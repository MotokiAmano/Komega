% Generated by Sphinx.
\def\sphinxdocclass{report}
\documentclass[letterpaper,10pt,dvipdfmx,openany]{sphinxmanual}
\usepackage[utf8]{inputenc}
\DeclareUnicodeCharacter{00A0}{\nobreakspace}
\usepackage{cmap}
\usepackage[T1]{fontenc}

\usepackage{times}

\usepackage{longtable}
\usepackage{sphinx}
\usepackage{multirow}
\usepackage{pxjahyper}

\title{$K\omega$ マニュアル}
\date{2017 年 01 月 27 日}
\release{0.2}
\author{}
\newcommand{\sphinxlogo}{}
\renewcommand{\releasename}{リリース}
\makeindex

\makeatletter
\def\PYG@reset{\let\PYG@it=\relax \let\PYG@bf=\relax%
    \let\PYG@ul=\relax \let\PYG@tc=\relax%
    \let\PYG@bc=\relax \let\PYG@ff=\relax}
\def\PYG@tok#1{\csname PYG@tok@#1\endcsname}
\def\PYG@toks#1+{\ifx\relax#1\empty\else%
    \PYG@tok{#1}\expandafter\PYG@toks\fi}
\def\PYG@do#1{\PYG@bc{\PYG@tc{\PYG@ul{%
    \PYG@it{\PYG@bf{\PYG@ff{#1}}}}}}}
\def\PYG#1#2{\PYG@reset\PYG@toks#1+\relax+\PYG@do{#2}}

\expandafter\def\csname PYG@tok@gd\endcsname{\def\PYG@tc##1{\textcolor[rgb]{0.63,0.00,0.00}{##1}}}
\expandafter\def\csname PYG@tok@gu\endcsname{\let\PYG@bf=\textbf\def\PYG@tc##1{\textcolor[rgb]{0.50,0.00,0.50}{##1}}}
\expandafter\def\csname PYG@tok@gt\endcsname{\def\PYG@tc##1{\textcolor[rgb]{0.00,0.27,0.87}{##1}}}
\expandafter\def\csname PYG@tok@gs\endcsname{\let\PYG@bf=\textbf}
\expandafter\def\csname PYG@tok@gr\endcsname{\def\PYG@tc##1{\textcolor[rgb]{1.00,0.00,0.00}{##1}}}
\expandafter\def\csname PYG@tok@cm\endcsname{\let\PYG@it=\textit\def\PYG@tc##1{\textcolor[rgb]{0.25,0.50,0.56}{##1}}}
\expandafter\def\csname PYG@tok@vg\endcsname{\def\PYG@tc##1{\textcolor[rgb]{0.73,0.38,0.84}{##1}}}
\expandafter\def\csname PYG@tok@m\endcsname{\def\PYG@tc##1{\textcolor[rgb]{0.13,0.50,0.31}{##1}}}
\expandafter\def\csname PYG@tok@mh\endcsname{\def\PYG@tc##1{\textcolor[rgb]{0.13,0.50,0.31}{##1}}}
\expandafter\def\csname PYG@tok@cs\endcsname{\def\PYG@tc##1{\textcolor[rgb]{0.25,0.50,0.56}{##1}}\def\PYG@bc##1{\setlength{\fboxsep}{0pt}\colorbox[rgb]{1.00,0.94,0.94}{\strut ##1}}}
\expandafter\def\csname PYG@tok@ge\endcsname{\let\PYG@it=\textit}
\expandafter\def\csname PYG@tok@vc\endcsname{\def\PYG@tc##1{\textcolor[rgb]{0.73,0.38,0.84}{##1}}}
\expandafter\def\csname PYG@tok@il\endcsname{\def\PYG@tc##1{\textcolor[rgb]{0.13,0.50,0.31}{##1}}}
\expandafter\def\csname PYG@tok@go\endcsname{\def\PYG@tc##1{\textcolor[rgb]{0.20,0.20,0.20}{##1}}}
\expandafter\def\csname PYG@tok@cp\endcsname{\def\PYG@tc##1{\textcolor[rgb]{0.00,0.44,0.13}{##1}}}
\expandafter\def\csname PYG@tok@gi\endcsname{\def\PYG@tc##1{\textcolor[rgb]{0.00,0.63,0.00}{##1}}}
\expandafter\def\csname PYG@tok@gh\endcsname{\let\PYG@bf=\textbf\def\PYG@tc##1{\textcolor[rgb]{0.00,0.00,0.50}{##1}}}
\expandafter\def\csname PYG@tok@ni\endcsname{\let\PYG@bf=\textbf\def\PYG@tc##1{\textcolor[rgb]{0.84,0.33,0.22}{##1}}}
\expandafter\def\csname PYG@tok@nl\endcsname{\let\PYG@bf=\textbf\def\PYG@tc##1{\textcolor[rgb]{0.00,0.13,0.44}{##1}}}
\expandafter\def\csname PYG@tok@nn\endcsname{\let\PYG@bf=\textbf\def\PYG@tc##1{\textcolor[rgb]{0.05,0.52,0.71}{##1}}}
\expandafter\def\csname PYG@tok@no\endcsname{\def\PYG@tc##1{\textcolor[rgb]{0.38,0.68,0.84}{##1}}}
\expandafter\def\csname PYG@tok@na\endcsname{\def\PYG@tc##1{\textcolor[rgb]{0.25,0.44,0.63}{##1}}}
\expandafter\def\csname PYG@tok@nb\endcsname{\def\PYG@tc##1{\textcolor[rgb]{0.00,0.44,0.13}{##1}}}
\expandafter\def\csname PYG@tok@nc\endcsname{\let\PYG@bf=\textbf\def\PYG@tc##1{\textcolor[rgb]{0.05,0.52,0.71}{##1}}}
\expandafter\def\csname PYG@tok@nd\endcsname{\let\PYG@bf=\textbf\def\PYG@tc##1{\textcolor[rgb]{0.33,0.33,0.33}{##1}}}
\expandafter\def\csname PYG@tok@ne\endcsname{\def\PYG@tc##1{\textcolor[rgb]{0.00,0.44,0.13}{##1}}}
\expandafter\def\csname PYG@tok@nf\endcsname{\def\PYG@tc##1{\textcolor[rgb]{0.02,0.16,0.49}{##1}}}
\expandafter\def\csname PYG@tok@si\endcsname{\let\PYG@it=\textit\def\PYG@tc##1{\textcolor[rgb]{0.44,0.63,0.82}{##1}}}
\expandafter\def\csname PYG@tok@s2\endcsname{\def\PYG@tc##1{\textcolor[rgb]{0.25,0.44,0.63}{##1}}}
\expandafter\def\csname PYG@tok@vi\endcsname{\def\PYG@tc##1{\textcolor[rgb]{0.73,0.38,0.84}{##1}}}
\expandafter\def\csname PYG@tok@nt\endcsname{\let\PYG@bf=\textbf\def\PYG@tc##1{\textcolor[rgb]{0.02,0.16,0.45}{##1}}}
\expandafter\def\csname PYG@tok@nv\endcsname{\def\PYG@tc##1{\textcolor[rgb]{0.73,0.38,0.84}{##1}}}
\expandafter\def\csname PYG@tok@s1\endcsname{\def\PYG@tc##1{\textcolor[rgb]{0.25,0.44,0.63}{##1}}}
\expandafter\def\csname PYG@tok@gp\endcsname{\let\PYG@bf=\textbf\def\PYG@tc##1{\textcolor[rgb]{0.78,0.36,0.04}{##1}}}
\expandafter\def\csname PYG@tok@sh\endcsname{\def\PYG@tc##1{\textcolor[rgb]{0.25,0.44,0.63}{##1}}}
\expandafter\def\csname PYG@tok@ow\endcsname{\let\PYG@bf=\textbf\def\PYG@tc##1{\textcolor[rgb]{0.00,0.44,0.13}{##1}}}
\expandafter\def\csname PYG@tok@sx\endcsname{\def\PYG@tc##1{\textcolor[rgb]{0.78,0.36,0.04}{##1}}}
\expandafter\def\csname PYG@tok@bp\endcsname{\def\PYG@tc##1{\textcolor[rgb]{0.00,0.44,0.13}{##1}}}
\expandafter\def\csname PYG@tok@c1\endcsname{\let\PYG@it=\textit\def\PYG@tc##1{\textcolor[rgb]{0.25,0.50,0.56}{##1}}}
\expandafter\def\csname PYG@tok@kc\endcsname{\let\PYG@bf=\textbf\def\PYG@tc##1{\textcolor[rgb]{0.00,0.44,0.13}{##1}}}
\expandafter\def\csname PYG@tok@c\endcsname{\let\PYG@it=\textit\def\PYG@tc##1{\textcolor[rgb]{0.25,0.50,0.56}{##1}}}
\expandafter\def\csname PYG@tok@mf\endcsname{\def\PYG@tc##1{\textcolor[rgb]{0.13,0.50,0.31}{##1}}}
\expandafter\def\csname PYG@tok@err\endcsname{\def\PYG@bc##1{\setlength{\fboxsep}{0pt}\fcolorbox[rgb]{1.00,0.00,0.00}{1,1,1}{\strut ##1}}}
\expandafter\def\csname PYG@tok@kd\endcsname{\let\PYG@bf=\textbf\def\PYG@tc##1{\textcolor[rgb]{0.00,0.44,0.13}{##1}}}
\expandafter\def\csname PYG@tok@ss\endcsname{\def\PYG@tc##1{\textcolor[rgb]{0.32,0.47,0.09}{##1}}}
\expandafter\def\csname PYG@tok@sr\endcsname{\def\PYG@tc##1{\textcolor[rgb]{0.14,0.33,0.53}{##1}}}
\expandafter\def\csname PYG@tok@mo\endcsname{\def\PYG@tc##1{\textcolor[rgb]{0.13,0.50,0.31}{##1}}}
\expandafter\def\csname PYG@tok@mi\endcsname{\def\PYG@tc##1{\textcolor[rgb]{0.13,0.50,0.31}{##1}}}
\expandafter\def\csname PYG@tok@kn\endcsname{\let\PYG@bf=\textbf\def\PYG@tc##1{\textcolor[rgb]{0.00,0.44,0.13}{##1}}}
\expandafter\def\csname PYG@tok@o\endcsname{\def\PYG@tc##1{\textcolor[rgb]{0.40,0.40,0.40}{##1}}}
\expandafter\def\csname PYG@tok@kr\endcsname{\let\PYG@bf=\textbf\def\PYG@tc##1{\textcolor[rgb]{0.00,0.44,0.13}{##1}}}
\expandafter\def\csname PYG@tok@s\endcsname{\def\PYG@tc##1{\textcolor[rgb]{0.25,0.44,0.63}{##1}}}
\expandafter\def\csname PYG@tok@kp\endcsname{\def\PYG@tc##1{\textcolor[rgb]{0.00,0.44,0.13}{##1}}}
\expandafter\def\csname PYG@tok@w\endcsname{\def\PYG@tc##1{\textcolor[rgb]{0.73,0.73,0.73}{##1}}}
\expandafter\def\csname PYG@tok@kt\endcsname{\def\PYG@tc##1{\textcolor[rgb]{0.56,0.13,0.00}{##1}}}
\expandafter\def\csname PYG@tok@sc\endcsname{\def\PYG@tc##1{\textcolor[rgb]{0.25,0.44,0.63}{##1}}}
\expandafter\def\csname PYG@tok@sb\endcsname{\def\PYG@tc##1{\textcolor[rgb]{0.25,0.44,0.63}{##1}}}
\expandafter\def\csname PYG@tok@k\endcsname{\let\PYG@bf=\textbf\def\PYG@tc##1{\textcolor[rgb]{0.00,0.44,0.13}{##1}}}
\expandafter\def\csname PYG@tok@se\endcsname{\let\PYG@bf=\textbf\def\PYG@tc##1{\textcolor[rgb]{0.25,0.44,0.63}{##1}}}
\expandafter\def\csname PYG@tok@sd\endcsname{\let\PYG@it=\textit\def\PYG@tc##1{\textcolor[rgb]{0.25,0.44,0.63}{##1}}}

\def\PYGZbs{\char`\\}
\def\PYGZus{\char`\_}
\def\PYGZob{\char`\{}
\def\PYGZcb{\char`\}}
\def\PYGZca{\char`\^}
\def\PYGZam{\char`\&}
\def\PYGZlt{\char`\<}
\def\PYGZgt{\char`\>}
\def\PYGZsh{\char`\#}
\def\PYGZpc{\char`\%}
\def\PYGZdl{\char`\$}
\def\PYGZhy{\char`\-}
\def\PYGZsq{\char`\'}
\def\PYGZdq{\char`\"}
\def\PYGZti{\char`\~}
% for compatibility with earlier versions
\def\PYGZat{@}
\def\PYGZlb{[}
\def\PYGZrb{]}
\makeatother

\begin{document}

\maketitle
\tableofcontents
\phantomsection\label{index::doc}



\chapter{概要}
\label{komega_overview_ja::doc}\label{komega_overview_ja:welcome-to-s-documentation}\label{komega_overview_ja:id1}
本資料はISSP Math
Libraryの内の、Krylov部分空間法に基づくシフト線形方程式群ソルバーライブラリ
\(K(\omega)\)に関するマニュアルである. 本ライブラリは,
(射影付き)シフト線形問題
\begin{gather}
\begin{split}\begin{align}
  G_{i j}(z) = \langle i | (z {\hat I} -{\hat H})^{-1}| j \rangle \equiv
  {\boldsymbol \varphi}_i^{*} \cdot (z{\hat I}-{\hat H})^{-1} {\boldsymbol \varphi}_j
  \end{align}\end{split}\notag
\end{gather}
を, Krylov部分空間法を用いて解くためのルーチンを提供する.
言語はfortranを用いる. また, BLASレベル1ルーチンを使用する.


\chapter{ライセンス}
\label{komega_copyright_ja::doc}\label{komega_copyright_ja:id1}
\emph{© 2016- The University of Tokyo. All rights reserved.}

\begin{DUlineblock}{0em}
\item[] This software is developed under the support of
\item[] ``\emph{Project for advancement of software usability in materials science}'' by The
\item[] Institute for Solid State Physics, The University of Tokyo.
\item[] 
\item[] This library is free software; you can redistribute it and/or
\item[] modify it under the terms of the GNU Lesser General Public
\item[] License as published by the Free Software Foundation; either
\item[] version 2.1 of the License, or (at your option) any later version.
\item[] This library is distributed in the hope that it will be useful,
\item[] but WITHOUT ANY WARRANTY; without even the implied warranty of
\item[] MERCHANTABILITY or FITNESS FOR A PARTICULAR PURPOSE. See the GNU
\item[] Lesser General Public License for more details.
\item[] 
\item[] You should have received a copy of the GNU Lesser General Public
\item[] License along with this library; if not, write to the Free Software
\item[] Foundation, Inc., 59 Temple Place, Suite 330, Boston, MA 02111-1307 USA
\item[] 
\item[] For more details, See `COPYING.LESSER' in the root directory of this library.
\end{DUlineblock}


\chapter{アルゴリズム}
\label{komega_algorithm_ja::doc}\label{komega_algorithm_ja:id1}
このライブラリは,
\({\hat H}\) および \(z\) が複素数であるか実数であるかに応じて,
次の4種類の計算をサポートする( \({\hat H}\) は複素数の場合はエルミート行列,
実数の場合は実対称行列).
\begin{itemize}
\item {} 
\({\hat H}\) も \(z\) も両方複素数の場合 : Shifted
Bi-Conjugate Gradient(BiCG)法 {\hyperref[komega_ref_ja:ref]{\emph{{[}1{]}}}}

\item {} 
\({\hat H}\) が実数で \(z\) が複素数の場合 : Shifted
Conjugate Orthogonal Conjugate Gradient(COCG)法 {\hyperref[komega_ref_ja:ref]{\emph{{[}2{]}}}}

\item {} 
\({\hat H}\) が複素数で \(z\) が実数の場合 : Shifted
Conjugate Gradient(CG)法 (複素ベクトル)

\item {} 
\({\hat H}\) も \(z\) も両方実数の場合 : Shifted Conjugate
Gradient(CG)法 (実ベクトル)

\end{itemize}

いずれの場合も Seed switching {\hyperref[komega_ref_ja:ref]{\emph{{[}2{]}}}} を行う. 左ベクトルが \(N_L\) 個,
右ベクトルが \(N_R\) 個(典型的には1個)あるとする. 以下,
各手法のアルゴリズムを記載する.


\section{Seed switch 付き Shifted BiCG法}
\label{komega_algorithm_ja:seed-switch-shifted-bicg}
\(G_{i j}(z_k) = 0 (i=1 \cdots N_L,\; j = 1 \cdots N_R,\; k=1 \cdots N_z)\)

do \(j = 1 \cdots N_R\)
\begin{quote}

\({\boldsymbol r} = {\boldsymbol \varphi_j}\),

\({\tilde {\boldsymbol r}} =\) 任意,
\({\boldsymbol r}^{\rm old} = {\tilde {\boldsymbol r}}^{\rm old} = {\bf 0}\)

\(p_{i k} = 0(i=1 \cdots N_L,\; k=1 \cdots N_z),\; \pi_k=\pi_k^{\rm old} = 1(k=1 \cdots N_z)\)

\(\rho = \infty,\; \alpha = 1,\; z_{\rm seed}=0\)

do iteration
\begin{quote}

\(\circ\) シード方程式

\(\rho^{\rm old} = \rho,\; \rho = {\tilde {\boldsymbol r}}^* \cdot {\boldsymbol r}\)

\(\beta = \rho / \rho^{\rm old}\)

\({\boldsymbol q} = (z_{\rm seed} {\hat I} - {\hat H}){\boldsymbol r}\)

\(\alpha^{\rm old} = \alpha,\; \alpha = \frac{\rho}{{\tilde {\boldsymbol r}}^*\cdot{\boldsymbol q} - \beta \rho / \alpha }\)

\(\circ\) シフト方程式

do \(k = 1 \cdots N_z\)
\begin{quote}

\(\pi_k^{\rm new} = [1+\alpha(z_k-z_{\rm seed})]\pi_k - \frac{\alpha \beta}{\alpha^{\rm old}}(\pi_k^{\rm old} - \pi_k)\)

do \(i = 1 \cdots N_L\)
\begin{quote}

\(p_{i k} = \frac{1}{\pi_k} {\boldsymbol \varphi}_i^* \cdot {\boldsymbol r} + \frac{\pi^{\rm old}_k \pi^{\rm old}_k}{\pi_k \pi_k} \beta p_{i k}\)

\(G_{i j}(z_k) = G_{i j}(z_k) + \frac{\pi_k}{\pi_k^{\rm new}} \alpha p_{i k}\)

\(\pi_k^{\rm old} = \pi_k\), \(\pi_k = \pi_k^{\rm new}\)
\end{quote}

end do \(i\)
\end{quote}

end do \(k\)

\({\boldsymbol q} = \left( 1 + \frac{\alpha \beta}{\alpha^{\rm old}} \right) {\boldsymbol r} - \alpha {\boldsymbol q} - \frac{\alpha \beta}{\alpha^{\rm old}} {\boldsymbol r}^{\rm old},\; {\boldsymbol r}^{\rm old} = {\boldsymbol r},\; {\boldsymbol r} = {\boldsymbol q}\)

\({\boldsymbol q} = (z_{\rm seed}^* {\hat I} - {\hat H}) {\tilde {\boldsymbol r}},\; {\boldsymbol q} = \left( 1 + \frac{\alpha^* \beta^*}{\alpha^{{\rm old}*}} \right) {\tilde {\boldsymbol r}} - \alpha^* {\boldsymbol q} - \frac{\alpha^* \beta^*}{\alpha^{{\rm old} *}} {\tilde {\boldsymbol r}}^{\rm old},\; {\tilde {\boldsymbol r}}^{\rm old} = {\tilde {\boldsymbol r}},\; {\tilde {\boldsymbol r}} = {\boldsymbol q}\)

\(\circ\) Seed switch

\(|\pi_k|\) が最も小さい \(k\) を探す. \(\rightarrow z_{\rm seed},\; \pi_{\rm seed},\; \pi_{\rm seed}^{\rm old}\)

\({\boldsymbol r} = {\boldsymbol r} / \pi_{\rm seed},\; {\boldsymbol r}^{\rm old} = {\boldsymbol r}^{\rm old} / \pi_{\rm seed}^{\rm old},\; {\tilde {\boldsymbol r}} = {\tilde {\boldsymbol r}} / \pi_{\rm seed}^*,\; {\tilde {\boldsymbol r}}^{\rm old} = {\tilde {\boldsymbol r}}^{\rm old} / \pi_{\rm seed}^{{\rm old}*}\)

\(\alpha = (\pi_{\rm seed}^{\rm old} / \pi_{\rm seed}) \alpha\), \(\rho = \rho / (\pi_{\rm seed}^{\rm old} \pi_{\rm seed}^{\rm old})\)

\(\{\pi_k = \pi_k / \pi_{\rm seed},\; \pi_k^{\rm old} = \pi_k^{\rm old} / \pi_{\rm seed}^{\rm old}\}\)

if( \(|{\boldsymbol r}| <\) Threshold) exit
\end{quote}

end do iteration
\end{quote}

end do \(j\)


\section{Seed switch 付き Shifted COCG法}
\label{komega_algorithm_ja:seed-switch-shifted-cocg}
BiCGのアルゴリズムで,
\({\tilde {\boldsymbol r}} = {\boldsymbol r}^*,\; {\tilde {\boldsymbol r}}^{\rm old} = {\boldsymbol r}^{{\rm old}*}\) とすると得られる.

\(G_{i j}(z_k) = 0 (i=1 \cdots N_L,\; j = 1 \cdots N_R,\; k=1 \cdots N_z)\)

do \(j = 1 \cdots N_R\)
\begin{quote}

\({\boldsymbol r} = {\boldsymbol \varphi_j}\), \({\boldsymbol r}^{\rm old} = {\bf 0}\)

\(p_{i k} = 0(i=1 \cdots N_L,\; k=1 \cdots N_z),\; \pi_k=\pi_k^{\rm old} = 1(k=1 \cdots N_z)\)

\(\rho = \infty,\; \alpha = 1,\; z_{\rm seed}=0\)

do iteration
\begin{quote}

\(\circ\) シード方程式

\(\rho^{\rm old} = \rho,\; \rho = {\boldsymbol r} \cdot {\boldsymbol r}\)

\(\beta = \rho / \rho^{\rm old}\)

\({\boldsymbol q} = (z_{\rm seed} {\hat I} - {\hat H}){\boldsymbol r}\)

\(\alpha^{\rm old} = \alpha,\; \alpha = \frac{\rho}{{\boldsymbol r}\cdot{\boldsymbol q} - \beta \rho / \alpha }\)

\(\circ\) シフト方程式

do \(k = 1 \cdots N_z\)
\begin{quote}

\(\pi_k^{\rm new} = [1+\alpha(z_k-z_{\rm seed})]\pi_k - \frac{\alpha \beta}{\alpha^{\rm old}}(\pi_k^{\rm old} - \pi_k)\)

do \(i = 1 \cdots N_L\)
\begin{quote}

\(p_{i k} = \frac{1}{\pi_k} {\boldsymbol \varphi}_i^* \cdot {\boldsymbol r} + \frac{\pi^{\rm old}_k \pi^{\rm old}_k}{\pi_k \pi_k} \beta p_{i k}\)

\(G_{i j}(z_k) = G_{i j}(z_k) + \frac{\pi_k}{\pi_k^{\rm new}} \alpha p_{i k}\)

\(\pi_k^{\rm old} = \pi_k\), \(\pi_k = \pi_k^{\rm new}\)
\end{quote}

end do \(i\)
\end{quote}

end do \(k\)

\({\boldsymbol q} = \left( 1 + \frac{\alpha \beta}{\alpha^{\rm old}} \right) {\boldsymbol r} - \alpha {\boldsymbol q} - \frac{\alpha \beta}{\alpha^{\rm old}} {\boldsymbol r}^{\rm old},\; {\boldsymbol r}^{\rm old} = {\boldsymbol r},\; {\boldsymbol r} = {\boldsymbol q}\)

\(\circ\) Seed switch

\(|\pi_k|\) が最も小さい \(k\) を探す. \(\rightarrow z_{\rm seed},\; \pi_{\rm seed},\; \pi_{\rm seed}^{\rm old}\)

\({\boldsymbol r} = {\boldsymbol r} / \pi_{\rm seed},\; {\boldsymbol r}^{\rm old} = {\boldsymbol r}^{\rm old} / \pi_{\rm seed}^{\rm old}\)

\(\alpha = (\pi_{\rm seed}^{\rm old} / \pi_{\rm seed}) \alpha\), \(\rho = \rho / (\pi_{\rm seed}^{\rm old} \pi_{\rm seed}^{\rm old})\)

\(\{\pi_k = \pi_k/\pi_{\rm seed},\; \pi_k^{\rm old} = \pi_k^{\rm old} / \pi_{\rm seed}^{\rm old}\}\)

if( \(|{\boldsymbol r}| <\) Threshold) exit
\end{quote}

end do iteration
\end{quote}

end do \(j\)


\section{Seed switch 付き Shifted CG法}
\label{komega_algorithm_ja:seed-switch-shifted-cg}
BiCGのアルゴリズムで,
\({\tilde {\boldsymbol r}} = {\boldsymbol r},\; {\tilde {\boldsymbol r}}^{\rm old} = {\boldsymbol r}^{\rm old}\) とすると得られる.

\(G_{i j}(z_k) = 0 (i=1 \cdots N_L,\; j = 1 \cdots N_R,\; k=1 \cdots N_z)\)

do \(j = 1 \cdots N_R\)
\begin{quote}

\({\boldsymbol r} = {\boldsymbol \varphi_j}\), \({\boldsymbol r}^{\rm old} = {\bf 0}\)

\(p_{i k} = 0(i=1 \cdots N_L,\; k=1 \cdots N_z),\; \pi_k=\pi_k^{\rm old} = 1(k=1 \cdots N_z)\)

\(\rho = \infty,\; \alpha = 1,\; z_{\rm seed}=0\)

do iteration
\begin{quote}

\(\circ\) シード方程式

\(\rho^{\rm old} = \rho,\; \rho = {\boldsymbol r}^* \cdot {\boldsymbol r}\)

\(\beta = \rho / \rho^{\rm old}\)

\({\boldsymbol q} = (z_{\rm seed} {\hat I} - {\hat H}){\boldsymbol r}\)

\(\alpha^{\rm old} = \alpha,\; \alpha = \frac{\rho}{{\boldsymbol r}^* \cdot {\boldsymbol q} - \beta \rho / \alpha }\)

\(\circ\) シフト方程式

do \(k = 1 \cdots N_z\)
\begin{quote}

\(\pi_k^{\rm new} = [1+\alpha(z_k-z_{\rm seed})]\pi_k - \frac{\alpha \beta}{\alpha^{\rm old}}(\pi_k^{\rm old} - \pi_k)\)

do \(i = 1 \cdots N_L\)
\begin{quote}

\(p_{i k} = \frac{1}{\pi_k} {\boldsymbol \varphi}_i^* \cdot {\boldsymbol r} + \left(\frac{\pi^{\rm old}_k}{\pi_k } \right)^2 \beta p_{i k}\)

\(G_{i j}(z_k) = G_{i j}(z_k) + \frac{\pi_k}{\pi_k^{\rm new}} \alpha p_{i k}\)

\(\pi_k^{\rm old} = \pi_k\), \(\pi_k = \pi_k^{\rm new}\)
\end{quote}

end do \(i\)
\end{quote}

end do \(k\)

\({\boldsymbol q} = \left( 1 + \frac{\alpha \beta}{\alpha^{\rm old}} \right) {\boldsymbol r} - \alpha {\boldsymbol q} - \frac{\alpha \beta}{\alpha^{\rm old}} {\boldsymbol r}^{\rm old},\; {\boldsymbol r}^{\rm old} = {\boldsymbol r},\; {\boldsymbol r} = {\boldsymbol q}\)

\(\circ\) Seed switch

\(|\pi_k|\) が最も小さい \(k\) を探す. \(\rightarrow z_{\rm seed},\; \pi_{\rm seed},\; \pi_{\rm seed}^{\rm old}\)

\({\boldsymbol r} = {\boldsymbol r} / \pi_{\rm seed},\; {\boldsymbol r}^{\rm old} = {\boldsymbol r}^{\rm old} / \pi_{\rm seed}^{\rm old}\)

\(\alpha = (\pi_{\rm seed}^{\rm old} / \pi_{\rm seed}) \alpha\), \(\rho = \rho / {\pi_{\rm seed}^{\rm old}}^2\)

\(\{\pi_k = \pi_k/\pi_{\rm seed},\; \pi_k^{\rm old} = \pi_k^{\rm old}/\pi_{\rm seed}^{\rm old}\}\)

if( \(|{\boldsymbol r}| <\) Threshold) exit
\end{quote}

end do iteration
\end{quote}

end do \(j\)


\chapter{プログラム内でのライブラリの動作イメージ}
\label{komega_workflow_ja::doc}\label{komega_workflow_ja:id1}
以下では \(N_R\) のループは省略する(各右辺ベクトルごとに同じ事をすればいいので).
また \(G_{i j}(z_k)\) の代わりに \(N_z\) 個の \(N_L\) 次元の解ベクトル \({\bf x}_{k}\) を求める.

ライブラリの各ルーチンの名前は次の通りである.
\begin{itemize}
\item {} 
\code{komega\_bicg\_init}, \code{komega\_cocg\_init}, \code{komega\_cg\_c\_init},
\code{komega\_cg\_r\_init}

ライブラリ内部で使う(ユーザーの目に触れない)変数のAllocateや初期値設定を行う.

\item {} 
\code{komega\_bicg\_update}, \code{komega\_cocg\_update},
\code{komega\_cg\_c\_update}, \code{komega\_cg\_r\_update}

Iteration の中で呼び出される. 解ベクトル群の更新等を行う.

\item {} 
\code{komega\_bicg\_finalize}, \code{komega\_cocg\_finalize},
\code{komega\_cg\_c\_finalize}, \code{komega\_cg\_r\_finalize}

Allocateしたライブラリ内部ベクトルを開放する.

\item {} 
\code{komega\_bicg\_getcoef}, \code{komega\_cocg\_getcoef},
\code{komega\_cg\_c\_getcoef}, \code{komega\_cg\_r\_getcoef}

各iterationで保存しておいた \(\alpha\), \(\beta\),
\(z_{\rm seed}\), \({\bf r}^{\rm L}\) を取り出す.

\item {} 
\code{komega\_bicg\_getvec}, \code{komega\_cocg\_getvec},
\code{komega\_cg\_c\_getvec}, \code{komega\_cg\_r\_getvec}

\({\boldsymbol r}\), \({\boldsymbol r}^{\rm old}\),
\({\tilde {\boldsymbol r}}\),
\({\tilde {\boldsymbol r}}^{\rm old}\) を取り出す.

\item {} 
\code{komega\_bicg\_restart}, \code{komega\_cocg\_restart},
\code{komega\_cg\_c\_restart}, \code{komega\_cg\_r\_restart}

保存しておいた \(\alpha\) 等を用いて,
新規の \(z\) での計算を行う.
\({\boldsymbol r}\) 等も有る場合には \code{komega\_bicg\_init},
\code{komega\_cocg\_init}, \code{komega\_cg\_c\_init}, \code{komega\_cg\_r\_init}
の代わりに用いてリスタートすることもできる.

\end{itemize}

\begin{notice}{note}{ノート:}\begin{itemize}
\item {} 
\code{komega\_*\_init} を呼び出す前にサイズ \(N_H\) のベクトルを2本
(BiCGの時には4本)Allocateしておく.

\item {} 
ハミルトニアン-ベクトル積を行う部分はあらかじめ作成しておく.

\item {} 
解ベクトルをAllocateしておく. ただし,
解ベクトルの長さは必ずしも \(N_H\) である必要はない.
実際前節の場合は \(N_L\) である.
この時(双)共役勾配ベクトル \({\bf p}_k\) も
\(N_z\) 本の \(N_L\) 次元のベクトルである.
ユーザーは \(N_H\) 次元の残差ベクトルを \(N_L\) 次元へ変換する
ルーチン/関数をあらかじめ作っておかなければならない.
\begin{gather}
\begin{split}\begin{aligned}
{\bf r}^{\rm L} = {\hat P}^\dagger {\boldsymbol r}, \qquad
{\hat P} \equiv ({\boldsymbol \varphi}_1, \cdots, {\boldsymbol \varphi}_{N_L})
\end{aligned}\end{split}\notag
\end{gather}
\item {} 
\code{komega\_*\_update} の出力 \code{status} の第一成分が負の値になった場合には,
解が収束した, もしくは破たんしたことを表す.
したがって \code{status(1) \textless{} 0} でループを抜けるようにしておく.

\item {} 
\code{komega\_*\_update} 内での収束判定には,
シード点での残差ベクトルの2-ノルムが使われる.
すなわち, すべてのシフト点での残差ベクトルの2-ノルムが
\code{threshold} を下回った時に収束したと見做される.

\item {} 
各反復での \(\alpha, \beta, {\bf r}^{\rm L}\) を保存しておき,
あとで再利用する場合には最大反復回数 \code{itermax} を \code{0} 以外の値に設定する.

\end{itemize}
\end{notice}


\section{Shifted BiCGライブラリの動作イメージ}
\label{komega_workflow_ja:shifted-bicg}
Allocate \({\boldsymbol v}_{1 2}\), \({\boldsymbol v}_{1 3}\),
\({\boldsymbol v}_2\), \({\boldsymbol v}_3\),
\(\{{\bf x}_k\}, {\bf r}^{\rm L}\)
\({\boldsymbol v}_2 = {\boldsymbol \varphi_j}\)

\code{komega\_bicg\_init(N\_H, N\_L, N\_z, x, z, itermax, threshold)} start
\begin{quote}

Allocate \({\boldsymbol v}_3\), \({\boldsymbol v}_5\),
\(\{\pi_k\}\), \(\{\pi_k^{\rm old}\}\), \(\{{\bf p}_k\}\)

Copy \(\{z_k\}\)

\code{itermax} \(\neq\) \code{0} ならば \(\alpha\),
\(\beta\), \({\bf r}^{\rm L}\) を保存する配列を確保する.

\({\boldsymbol v}_4 = {\boldsymbol v}_2^*\) (任意),
\({\boldsymbol v}_3 = {\boldsymbol v}_5 = {\bf 0}\),

\({\bf p}_{k} = {\bf x}_k = {\bf 0}(k=1 \cdots N_z),\; \pi_k=\pi_k^{\rm old} = 1(k=1 \cdots N_z)\)

\(\rho = \infty,\; \alpha = 1,\; z_{\rm seed}=0\)

( \({\boldsymbol v}_2 \equiv {\boldsymbol r}\),
\({\boldsymbol v}_3 \equiv {\boldsymbol r}^{\rm old}\),
\({\boldsymbol v}_4 \equiv {\tilde {\boldsymbol r}}\),
\({\boldsymbol v}_5 \equiv {\tilde {\boldsymbol r}}^{\rm old}\). )
\end{quote}

\code{komega\_bicg\_init} finish

do iteration
\begin{quote}

\({\bf r}^{\rm L} = {\hat P}^\dagger {\boldsymbol v}_2\)

\({\boldsymbol v}_{1 2} = {\hat H} {\boldsymbol v}_2\),
\({\boldsymbol v}_{1 4} = {\hat H} {\boldsymbol v}_4\)
{[} \(({\boldsymbol v}_{1 2}, {\boldsymbol v}_{1 4}) = {\hat H} ({\boldsymbol v}_2, {\boldsymbol v}_4)\) とも書ける{]}

\code{komega\_bicg\_update(v\_12, v\_2, v\_14, v\_4, x, r\_small, status)} start
\begin{quote}

\(\circ\) シード方程式

\(\rho^{\rm old} = \rho,\; \rho = {\boldsymbol v}_4^* \cdot {\boldsymbol v}_2\)

\(\beta = \rho / \rho^{\rm old}\)

\({\boldsymbol v}_{1 2} = z_{\rm seed} {\boldsymbol v}_2 - {\boldsymbol v}_{1 2}\),
\({\boldsymbol v}_{1 4} = z_{\rm seed}^* {\boldsymbol v}_4 - {\boldsymbol v}_{1 4}\)

\(\alpha^{\rm old} = \alpha,\; \alpha = \frac{\rho}{{\boldsymbol v}_3^* \cdot {\boldsymbol v}_{1 2} - \beta \rho / \alpha }\)

\(\circ\) シフト方程式

do \(k = 1 \cdots N_z\)
\begin{quote}

\(\pi_k^{\rm new} = [1+\alpha(z_k-z_{\rm seed})]\pi_k - \frac{\alpha \beta}{\alpha^{\rm old}}(\pi_k^{\rm old} - \pi_k)\)

\({\bf p}_{k} = \frac{1}{\pi_k} {\bf r}^{\rm L} + \frac{\pi^{\rm old}_k \pi^{\rm old}_k}{\pi_k \pi_k} \beta {\bf p}_{k}\)

\({\bf x}_{k} = {\bf x}_{k} + \frac{\pi_k}{\pi_k^{\rm new}} \alpha {\bf p}_{k}\)

\(\pi_k^{\rm old} = \pi_k\), \(\pi_k = \pi_k^{\rm new}\)
\end{quote}

end do \(k\)

\({\boldsymbol v}_{1 2} = \left( 1 + \frac{\alpha \beta}{\alpha^{\rm old}} \right) {\boldsymbol v}_2 - \alpha {\boldsymbol v}_{1 2} - \frac{\alpha \beta}{\alpha^{\rm old}} {\boldsymbol v}_3\),
\({\boldsymbol v}_3 = {\boldsymbol v}_2,\; {\boldsymbol v}_2 = {\boldsymbol v}_{1 2}\)

\({\boldsymbol v}_{1 4} = \left( 1 + \frac{\alpha^* \beta^*}{\alpha^{{\rm old}*}} \right) {\boldsymbol v}_4 - \alpha^* {\boldsymbol v}_{1 4} - \frac{\alpha^* \beta^*}{\alpha^{{\rm old} *}} {\boldsymbol v}_5\),
\({\boldsymbol v}_5 = {\boldsymbol v}_4,\; {\boldsymbol v}_4 = {\boldsymbol v}_{1 4}\)

\(\circ\) Seed switch

\(|\pi_k|\) が最も小さい \(k\) を探す.
\(\rightarrow z_{\rm seed},\; \pi_{\rm seed},\; \pi_{\rm seed}^{\rm old}\)

\({\boldsymbol v}_2 = {\boldsymbol v}_2 / \pi_{\rm seed}\),
\({\boldsymbol v}_3 = {\boldsymbol v}_3 / \pi_{\rm seed}^{\rm old}\),
\({\boldsymbol v}_4 = {\boldsymbol v}_4 / \pi_{\rm seed}^{*}\),
\({\boldsymbol v}_5 = {\boldsymbol v}_5 / \pi_{\rm seed}^{\rm old *}\)

\(\alpha = (\pi_{\rm seed}^{\rm old} / \pi_{\rm seed}) \alpha\),
\(\rho = \rho / (\pi_{\rm seed}^{\rm old} \pi_{\rm seed}^{\rm old})\)

\(\{\pi_k = \pi_k / \pi_{\rm seed},\; \pi_k^{\rm old} = \pi_k^{\rm old} / \pi_{\rm seed}^{\rm old}\}\)
\end{quote}

\code{komega\_bicg\_update} finish

if(status(1) \textless{} 0 (これは \(|{\boldsymbol v}_2| <\) Threshold となった事を意味する)) exit
\end{quote}

end do iteration

\code{komega\_bicg\_finalize} start
\begin{quote}

Deallocate \({\boldsymbol v}_4\), \({\boldsymbol v}_5\),
\(\{\pi_k\}\), \(\{\pi_k^{\rm old}\}\), \(\{{\bf p}_k\}\)
\end{quote}

\code{komega\_bicg\_finalize} finish


\section{Shifted COCGライブラリの動作イメージ}
\label{komega_workflow_ja:shifted-cocg}
Allocate \({\boldsymbol v}_1\), \({\boldsymbol v}_2\),
\(\{{\bf x}_k\}, {\bf r}^{\rm L}\)
\({\boldsymbol v}_2 = {\boldsymbol \varphi_j}\)

\code{komega\_cocg\_init(N\_H, N\_L, N\_z, x, z, itermax, threshold)} start
\begin{quote}

Allocate \({\boldsymbol v}_3\), \(\{\pi_k\}\),
\(\{\pi_k^{\rm old}\}\), \(\{{\bf p}_k\}\)

Copy \(\{z_k\}\)

\code{itermax} \(\neq\) \code{0} ならば \(\alpha\),
\(\beta\), \({\bf r}^{\rm L}\) を保存する配列を確保する.

\({\boldsymbol v}_3 = {\bf 0}\),

\({\bf p}_{k} = {\bf x}_k = {\bf 0}(k=1 \cdots N_z),\; \pi_k=\pi_k^{\rm old} = 1(k=1 \cdots N_z)\)

\(\rho = \infty,\; \alpha = 1,\; \beta=0,\; z_{\rm seed}=0\)

( \({\boldsymbol v}_2 \equiv {\boldsymbol r}\),
\({\boldsymbol v}_3 \equiv {\boldsymbol r}^{\rm old}\). )
\end{quote}

\code{komega\_cocg\_init} finish

do iteration
\begin{quote}

\({\bf r}^{\rm L} = {\hat P}^\dagger {\boldsymbol v}_2\)

\({\boldsymbol v}_1 = {\hat H} {\boldsymbol v}_2\)

\code{komega\_cocg\_update(v\_1, v\_2, x, r\_small, status)} start
\begin{quote}

\(\circ\) シード方程式

\(\rho^{\rm old} = \rho,\; \rho = {\boldsymbol v}_2 \cdot {\boldsymbol v}_2\)

\(\beta = \rho / \rho^{\rm old}\)

\({\boldsymbol v}_1 = z_{\rm seed} {\boldsymbol v}_2 - {\boldsymbol v}_1\)

\(\alpha^{\rm old} = \alpha,\; \alpha = \frac{\rho}{{\boldsymbol v}_2 \cdot {\boldsymbol v}_1 - \beta \rho / \alpha }\)

\(\circ\) シフト方程式

do \(k = 1 \cdots N_z\)
\begin{quote}

\(\pi_k^{\rm new} = [1+\alpha(z_k-z_{\rm seed})]\pi_k - \frac{\alpha \beta}{\alpha^{\rm old}}(\pi_k^{\rm old} - \pi_k)\)

\({\bf p}_{k} = \frac{1}{\pi_k} {\bf r}^{\rm L} + \frac{\pi^{\rm old}_k \pi^{\rm old}_k}{\pi_k \pi_k} \beta {\bf p}_{k}\)

\({\bf x}_{k} = {\bf x}_{k} + \frac{\pi_k}{\pi_k^{\rm new}} \alpha {\bf p}_{k}\)

\(\pi_k^{\rm old} = \pi_k,\; \pi_k = \pi_k^{\rm new}\)
\end{quote}

end do \(k\)

\({\boldsymbol v}_1 = \left( 1 + \frac{\alpha \beta}{\alpha^{\rm old}} \right) {\boldsymbol v}_2 - \alpha {\boldsymbol v}_1 - \frac{\alpha \beta}{\alpha^{\rm old}} {\boldsymbol v}_3\)

\({\boldsymbol v}_3 = {\boldsymbol v}_2\),
\({\boldsymbol v}_2 = {\boldsymbol v}_1\)

\(\circ\) Seed switch

\(|\pi_k|\) が最も小さい \(k\) を探す.
\(\rightarrow z_{\rm seed},\; \pi_{\rm seed},\; \pi_{\rm seed}^{\rm old}\)

\({\boldsymbol v}_2 = {\boldsymbol v}_2 / \pi_{\rm seed}\),
\({\boldsymbol v}_3 = {\boldsymbol v}_3 / \pi_{\rm seed}^{\rm old}\)

\(\alpha = (\pi_{\rm seed}^{\rm old} / \pi_{\rm seed}) \alpha\),
\(\rho = \rho / (\pi_{\rm seed}^{\rm old} \pi_{\rm seed}^{\rm old})\)

\(\{\pi_k = \pi_k / \pi_{\rm seed},\; \pi_k^{\rm old} = \pi_k^{\rm old} / \pi_{\rm seed}^{\rm old}\}\)
\end{quote}

\code{komega\_cocg\_update} finish

if(status(1) \textless{} 0 (これは \(|{\boldsymbol v}_2| <\) Threshold となった事を意味する)) exit
\end{quote}

end do iteration

\code{komega\_cocg\_finalize} start
\begin{quote}

Deallocate \({\boldsymbol v}_3\), \(\{\pi_k\}\),
\(\{\pi_k^{\rm old}\}\), \(\{{\bf p}_k\}\)
\end{quote}

\code{komega\_cocg\_finalize} finish


\section{Shifted CGライブラリの動作イメージ}
\label{komega_workflow_ja:shifted-cg}
COCGと同様.


\chapter{使用方法}
\label{komega_usage_ja::doc}\label{komega_usage_ja:id1}
各ライブラリともユーザーはライブラリ名および型を指定し,
\begin{itemize}
\item {} 
初期設定 ({\hyperref[komega_usage_ja:init]{\emph{*\_init}}})

\item {} 
アップデート ({\hyperref[komega_usage_ja:update]{\emph{*\_update}}})

\item {} 
(オプション) 再計算用の情報を取り出す. ({\hyperref[komega_usage_ja:getcoef]{\emph{*\_getcoef}}}, {\hyperref[komega_usage_ja:getvec]{\emph{*\_getvec}}})

\item {} 
終了関数 ({\hyperref[komega_usage_ja:finalize]{\emph{*\_finalize}}})

\end{itemize}

の手順で関数を使用することで, 計算が実施される. なお,
リスタートを行う場合には
\begin{itemize}
\item {} 
前回の計算で残した再計算用の情報を用いた初期設定({\hyperref[komega_usage_ja:restart]{\emph{*\_restart}}})

\item {} 
アップデート ({\hyperref[komega_usage_ja:update]{\emph{*\_update}}})

\item {} 
(オプション) 更なる再計算用の情報を取り出す. ({\hyperref[komega_usage_ja:getcoef]{\emph{*\_getcoef}}}, {\hyperref[komega_usage_ja:getvec]{\emph{*\_getvec}}})

\item {} 
終了関数 ({\hyperref[komega_usage_ja:finalize]{\emph{*\_finalize}}})

\end{itemize}

の手順で実行する.

\begin{notice}{warning}{警告:}
\(K\omega\) はスレッドセーフ \textbf{ではない} ので,
これらのルーチンは必ずOpenMPのパラレルリージョンの外から
呼ばなければならない.
\end{notice}

fortran から呼び出すときには

\begin{Verbatim}[commandchars=\\\{\}]
\PYG{k}{USE }\PYG{n+nv}{komega\PYGZus{}cg\PYGZus{}r} \PYG{c}{! 実ベクトルに対する共役勾配法}
\PYG{k}{USE }\PYG{n+nv}{komega\PYGZus{}cg\PYGZus{}c} \PYG{c}{! 複素ベクトルに対する共役勾配法}
\PYG{k}{USE }\PYG{n+nv}{komega\PYGZus{}cocg} \PYG{c}{! 共線直交共役勾配法}
\PYG{k}{USE }\PYG{n+nv}{komega\PYGZus{}bicg} \PYG{c}{! 双共役勾配法}
\end{Verbatim}

のようにモジュールを呼び出す(すべてのモジュールを呼び出す必要はなく,
行う計算の種類に対応するものだけでよい).
MPI/Hybrid並列版のルーチンを利用するときには,

\begin{Verbatim}[commandchars=\\\{\}]
\PYG{k}{USE }\PYG{n+nv}{pkomega\PYGZus{}cg\PYGZus{}r}
\PYG{k}{USE }\PYG{n+nv}{pkomega\PYGZus{}cg\PYGZus{}c}
\PYG{k}{USE }\PYG{n+nv}{pkomega\PYGZus{}cocg}
\PYG{k}{USE }\PYG{n+nv}{pkomega\PYGZus{}bicg}
\end{Verbatim}

のようにする.

C/C++で書かれたプログラムから呼び出すときには、

\begin{Verbatim}[commandchars=\\\{\}]
\PYG{c+cp}{\PYGZsh{}}\PYG{c+cp}{include komega\PYGZus{}cg\PYGZus{}r.h}
\PYG{c+cp}{\PYGZsh{}}\PYG{c+cp}{include komega\PYGZus{}cg\PYGZus{}c.h}
\PYG{c+cp}{\PYGZsh{}}\PYG{c+cp}{include komega\PYGZus{}cocg.h}
\PYG{c+cp}{\PYGZsh{}}\PYG{c+cp}{include komega\PYGZus{}bicg.h}
\end{Verbatim}

のようにヘッダーファイルを読み込む。
また、スカラー引数はすべてポインタとして渡す。
MPI/Hybrid並列版のルーチンを利用するときには,

\begin{Verbatim}[commandchars=\\\{\}]
\PYG{c+cp}{\PYGZsh{}}\PYG{c+cp}{include pkomega\PYGZus{}cg\PYGZus{}r.h}
\PYG{c+cp}{\PYGZsh{}}\PYG{c+cp}{include pkomega\PYGZus{}cg\PYGZus{}c.h}
\PYG{c+cp}{\PYGZsh{}}\PYG{c+cp}{include pkomega\PYGZus{}cocg.h}
\PYG{c+cp}{\PYGZsh{}}\PYG{c+cp}{include pkomega\PYGZus{}bicg.h}
\end{Verbatim}

のようにする。
またライブラリに渡すコミュニケーター変数を、次のようにC/C++のものからfortranのものに変換する。

\begin{Verbatim}[commandchars=\\\{\}]
\PYG{n}{comm\PYGZus{}f} \PYG{o}{=} \PYG{n}{MPI\PYGZus{}Comm\PYGZus{}c2f}\PYG{p}{(}\PYG{n}{comm\PYGZus{}c}\PYG{p}{)}\PYG{p}{;}
\end{Verbatim}


\section{各ルーチンの説明}
\label{komega_usage_ja:id2}

\subsection{*\_init}
\label{komega_usage_ja:init}\label{komega_usage_ja:id3}
ライブラリ内部変数の割り付けおよび初期化を行う.
シフト線形方程式を解く前に, 一番初めに実行する.

構文
\begin{quote}

Fortran シリアル/OpenMP版

\begin{Verbatim}[commandchars=\\\{\}]
\PYG{k}{CALL }\PYG{n+nv}{komega\PYGZus{}cg\PYGZus{}r\PYGZus{}init}\PYG{p}{(}\PYG{n+nv}{ndim}\PYG{p}{,} \PYG{n+nv}{nl}\PYG{p}{,} \PYG{n+nv}{nz}\PYG{p}{,} \PYG{n+nv}{x}\PYG{p}{,} \PYG{n+nv}{z}\PYG{p}{,} \PYG{n+nv}{itermax}\PYG{p}{,} \PYG{n+nv}{threshold}\PYG{p}{)}
\PYG{k}{CALL }\PYG{n+nv}{komega\PYGZus{}cg\PYGZus{}c\PYGZus{}init}\PYG{p}{(}\PYG{n+nv}{ndim}\PYG{p}{,} \PYG{n+nv}{nl}\PYG{p}{,} \PYG{n+nv}{nz}\PYG{p}{,} \PYG{n+nv}{x}\PYG{p}{,} \PYG{n+nv}{z}\PYG{p}{,} \PYG{n+nv}{itermax}\PYG{p}{,} \PYG{n+nv}{threshold}\PYG{p}{)}
\PYG{k}{CALL }\PYG{n+nv}{komega\PYGZus{}cocg\PYGZus{}init}\PYG{p}{(}\PYG{n+nv}{ndim}\PYG{p}{,} \PYG{n+nv}{nl}\PYG{p}{,} \PYG{n+nv}{nz}\PYG{p}{,} \PYG{n+nv}{x}\PYG{p}{,} \PYG{n+nv}{z}\PYG{p}{,} \PYG{n+nv}{itermax}\PYG{p}{,} \PYG{n+nv}{threshold}\PYG{p}{)}
\PYG{k}{CALL }\PYG{n+nv}{komega\PYGZus{}bicg\PYGZus{}init}\PYG{p}{(}\PYG{n+nv}{ndim}\PYG{p}{,} \PYG{n+nv}{nl}\PYG{p}{,} \PYG{n+nv}{nz}\PYG{p}{,} \PYG{n+nv}{x}\PYG{p}{,} \PYG{n+nv}{z}\PYG{p}{,} \PYG{n+nv}{itermax}\PYG{p}{,} \PYG{n+nv}{threshold}\PYG{p}{)}
\end{Verbatim}

Fortran MPI/Hybrid並列版

\begin{Verbatim}[commandchars=\\\{\}]
\PYG{k}{CALL }\PYG{n+nv}{pkomega\PYGZus{}cg\PYGZus{}r\PYGZus{}init}\PYG{p}{(}\PYG{n+nv}{ndim}\PYG{p}{,} \PYG{n+nv}{nl}\PYG{p}{,} \PYG{n+nv}{nz}\PYG{p}{,} \PYG{n+nv}{x}\PYG{p}{,} \PYG{n+nv}{z}\PYG{p}{,} \PYG{n+nv}{itermax}\PYG{p}{,} \PYG{n+nv}{threshold}\PYG{p}{,} \PYG{n+nv}{comm}\PYG{p}{)}
\PYG{k}{CALL }\PYG{n+nv}{pkomega\PYGZus{}cg\PYGZus{}c\PYGZus{}init}\PYG{p}{(}\PYG{n+nv}{ndim}\PYG{p}{,} \PYG{n+nv}{nl}\PYG{p}{,} \PYG{n+nv}{nz}\PYG{p}{,} \PYG{n+nv}{x}\PYG{p}{,} \PYG{n+nv}{z}\PYG{p}{,} \PYG{n+nv}{itermax}\PYG{p}{,} \PYG{n+nv}{threshold}\PYG{p}{,} \PYG{n+nv}{comm}\PYG{p}{)}
\PYG{k}{CALL }\PYG{n+nv}{pkomega\PYGZus{}cocg\PYGZus{}init}\PYG{p}{(}\PYG{n+nv}{ndim}\PYG{p}{,} \PYG{n+nv}{nl}\PYG{p}{,} \PYG{n+nv}{nz}\PYG{p}{,} \PYG{n+nv}{x}\PYG{p}{,} \PYG{n+nv}{z}\PYG{p}{,} \PYG{n+nv}{itermax}\PYG{p}{,} \PYG{n+nv}{threshold}\PYG{p}{,} \PYG{n+nv}{comm}\PYG{p}{)}
\PYG{k}{CALL }\PYG{n+nv}{pkomega\PYGZus{}bicg\PYGZus{}init}\PYG{p}{(}\PYG{n+nv}{ndim}\PYG{p}{,} \PYG{n+nv}{nl}\PYG{p}{,} \PYG{n+nv}{nz}\PYG{p}{,} \PYG{n+nv}{x}\PYG{p}{,} \PYG{n+nv}{z}\PYG{p}{,} \PYG{n+nv}{itermax}\PYG{p}{,} \PYG{n+nv}{threshold}\PYG{p}{,} \PYG{n+nv}{comm}\PYG{p}{)}
\end{Verbatim}

C/C++ シリアル/OpenMP版

\begin{Verbatim}[commandchars=\\\{\}]
\PYG{n}{komega\PYGZus{}cg\PYGZus{}r\PYGZus{}init}\PYG{p}{(}\PYG{o}{\PYGZam{}}\PYG{n}{ndim}\PYG{p}{,} \PYG{o}{\PYGZam{}}\PYG{n}{nl}\PYG{p}{,} \PYG{o}{\PYGZam{}}\PYG{n}{nz}\PYG{p}{,} \PYG{n}{x}\PYG{p}{,} \PYG{n}{z}\PYG{p}{,} \PYG{o}{\PYGZam{}}\PYG{n}{itermax}\PYG{p}{,} \PYG{o}{\PYGZam{}}\PYG{n}{threshold}\PYG{p}{)}\PYG{p}{;}
\PYG{n}{komega\PYGZus{}cg\PYGZus{}c\PYGZus{}init}\PYG{p}{(}\PYG{o}{\PYGZam{}}\PYG{n}{ndim}\PYG{p}{,} \PYG{o}{\PYGZam{}}\PYG{n}{nl}\PYG{p}{,} \PYG{o}{\PYGZam{}}\PYG{n}{nz}\PYG{p}{,} \PYG{n}{x}\PYG{p}{,} \PYG{n}{z}\PYG{p}{,} \PYG{o}{\PYGZam{}}\PYG{n}{itermax}\PYG{p}{,} \PYG{o}{\PYGZam{}}\PYG{n}{threshold}\PYG{p}{)}\PYG{p}{;}
\PYG{n}{komega\PYGZus{}cocg\PYGZus{}init}\PYG{p}{(}\PYG{o}{\PYGZam{}}\PYG{n}{ndim}\PYG{p}{,} \PYG{o}{\PYGZam{}}\PYG{n}{nl}\PYG{p}{,} \PYG{o}{\PYGZam{}}\PYG{n}{nz}\PYG{p}{,} \PYG{n}{x}\PYG{p}{,} \PYG{n}{z}\PYG{p}{,} \PYG{o}{\PYGZam{}}\PYG{n}{itermax}\PYG{p}{,} \PYG{o}{\PYGZam{}}\PYG{n}{threshold}\PYG{p}{)}\PYG{p}{;}
\PYG{n}{komega\PYGZus{}bicg\PYGZus{}init}\PYG{p}{(}\PYG{o}{\PYGZam{}}\PYG{n}{ndim}\PYG{p}{,} \PYG{o}{\PYGZam{}}\PYG{n}{nl}\PYG{p}{,} \PYG{o}{\PYGZam{}}\PYG{n}{nz}\PYG{p}{,} \PYG{n}{x}\PYG{p}{,} \PYG{n}{z}\PYG{p}{,} \PYG{o}{\PYGZam{}}\PYG{n}{itermax}\PYG{p}{,} \PYG{o}{\PYGZam{}}\PYG{n}{threshold}\PYG{p}{)}\PYG{p}{;}
\end{Verbatim}

C/C++ MPI/Hybrid並列版

\begin{Verbatim}[commandchars=\\\{\}]
\PYG{n}{pkomega\PYGZus{}cg\PYGZus{}r\PYGZus{}init}\PYG{p}{(}\PYG{o}{\PYGZam{}}\PYG{n}{ndim}\PYG{p}{,} \PYG{o}{\PYGZam{}}\PYG{n}{nl}\PYG{p}{,} \PYG{o}{\PYGZam{}}\PYG{n}{nz}\PYG{p}{,} \PYG{n}{x}\PYG{p}{,} \PYG{n}{z}\PYG{p}{,} \PYG{o}{\PYGZam{}}\PYG{n}{itermax}\PYG{p}{,} \PYG{o}{\PYGZam{}}\PYG{n}{threshold}\PYG{p}{,} \PYG{o}{\PYGZam{}}\PYG{n}{comm}\PYG{p}{)}\PYG{p}{;}
\PYG{n}{pkomega\PYGZus{}cg\PYGZus{}c\PYGZus{}init}\PYG{p}{(}\PYG{o}{\PYGZam{}}\PYG{n}{ndim}\PYG{p}{,} \PYG{o}{\PYGZam{}}\PYG{n}{nl}\PYG{p}{,} \PYG{o}{\PYGZam{}}\PYG{n}{nz}\PYG{p}{,} \PYG{n}{x}\PYG{p}{,} \PYG{n}{z}\PYG{p}{,} \PYG{o}{\PYGZam{}}\PYG{n}{itermax}\PYG{p}{,} \PYG{o}{\PYGZam{}}\PYG{n}{threshold}\PYG{p}{,} \PYG{o}{\PYGZam{}}\PYG{n}{comm}\PYG{p}{)}\PYG{p}{;}
\PYG{n}{pkomega\PYGZus{}cocg\PYGZus{}init}\PYG{p}{(}\PYG{o}{\PYGZam{}}\PYG{n}{ndim}\PYG{p}{,} \PYG{o}{\PYGZam{}}\PYG{n}{nl}\PYG{p}{,} \PYG{o}{\PYGZam{}}\PYG{n}{nz}\PYG{p}{,} \PYG{n}{x}\PYG{p}{,} \PYG{n}{z}\PYG{p}{,} \PYG{o}{\PYGZam{}}\PYG{n}{itermax}\PYG{p}{,} \PYG{o}{\PYGZam{}}\PYG{n}{threshold}\PYG{p}{,} \PYG{o}{\PYGZam{}}\PYG{n}{comm}\PYG{p}{)}\PYG{p}{;}
\PYG{n}{pkomega\PYGZus{}bicg\PYGZus{}init}\PYG{p}{(}\PYG{o}{\PYGZam{}}\PYG{n}{ndim}\PYG{p}{,} \PYG{o}{\PYGZam{}}\PYG{n}{nl}\PYG{p}{,} \PYG{o}{\PYGZam{}}\PYG{n}{nz}\PYG{p}{,} \PYG{n}{x}\PYG{p}{,} \PYG{n}{z}\PYG{p}{,} \PYG{o}{\PYGZam{}}\PYG{n}{itermax}\PYG{p}{,} \PYG{o}{\PYGZam{}}\PYG{n}{threshold}\PYG{p}{,} \PYG{o}{\PYGZam{}}\PYG{n}{comm}\PYG{p}{)}\PYG{p}{;}
\end{Verbatim}
\end{quote}

パラメーター
\begin{quote}

\begin{Verbatim}[commandchars=\\\{\}]
\PYG{k+kt}{INTEGER}\PYG{p}{,}\PYG{k}{INTENT}\PYG{p}{(}\PYG{n+nv}{IN}\PYG{p}{)} \PYG{k+kd}{::} \PYG{n+nv}{ndim}
\end{Verbatim}
\begin{quote}

線形方程式の次元.
以降のサブルーチンのパラメーターの次元で現れる \code{ndim} は
これと同じものになる.
\end{quote}

\begin{Verbatim}[commandchars=\\\{\}]
\PYG{k+kt}{INTEGER}\PYG{p}{,}\PYG{k}{INTENT}\PYG{p}{(}\PYG{n+nv}{IN}\PYG{p}{)} \PYG{k+kd}{::} \PYG{n+nv}{nl}
\end{Verbatim}
\begin{quote}

射影された解ベクトルの次元.
以降のサブルーチンのパラメーターの次元で現れる \code{nl} は
これと同じものになる.
\end{quote}

\begin{Verbatim}[commandchars=\\\{\}]
\PYG{k+kt}{INTEGER}\PYG{p}{,}\PYG{k}{INTENT}\PYG{p}{(}\PYG{n+nv}{IN}\PYG{p}{)} \PYG{k+kd}{::} \PYG{n+nv}{nz}
\end{Verbatim}
\begin{quote}

シフト点の数.
以降のサブルーチンのパラメーターの次元で現れる \code{nz} は
これと同じものになる.
\end{quote}

\begin{Verbatim}[commandchars=\\\{\}]
\PYG{k+kt}{REAL}\PYG{p}{(}\PYG{l+m+mi}{8}\PYG{p}{)}\PYG{p}{,}\PYG{k}{INTENT}\PYG{p}{(}\PYG{n+nv}{OUT}\PYG{p}{)} \PYG{k+kd}{::} \PYG{n+nv}{x}\PYG{p}{(}\PYG{n+nv}{nl}\PYG{o}{*}\PYG{n+nv}{nz}\PYG{p}{)} \PYG{c}{! (\PYGZdq{}CG\PYGZus{}R\PYGZus{}init\PYGZdq{}, \PYGZdq{}cg\PYGZus{}c\PYGZus{}init\PYGZdq{} の場合)}
\PYG{k+kt}{COMPLEX}\PYG{p}{(}\PYG{l+m+mi}{8}\PYG{p}{)}\PYG{p}{,}\PYG{k}{INTENT}\PYG{p}{(}\PYG{n+nv}{OUT}\PYG{p}{)} \PYG{k+kd}{::} \PYG{n+nv}{x}\PYG{p}{(}\PYG{n+nv}{nl}\PYG{o}{*}\PYG{n+nv}{nz}\PYG{p}{)} \PYG{c}{! (それ以外)}
\end{Verbatim}
\begin{quote}

解ベクトル. \code{0} ベクトルが返される.
\end{quote}

\begin{Verbatim}[commandchars=\\\{\}]
\PYG{k+kt}{REAL}\PYG{p}{(}\PYG{l+m+mi}{8}\PYG{p}{)}\PYG{p}{,}\PYG{k}{INTENT}\PYG{p}{(}\PYG{n+nv}{IN}\PYG{p}{)} \PYG{k+kd}{::} \PYG{n+nv}{z}\PYG{p}{(}\PYG{n+nv}{nz}\PYG{p}{)} \PYG{c}{! (\PYGZdq{}CG\PYGZus{}R\PYGZus{}init\PYGZdq{}, \PYGZdq{}cg\PYGZus{}c\PYGZus{}init\PYGZdq{} の場合)}
\PYG{k+kt}{COMPLEX}\PYG{p}{(}\PYG{l+m+mi}{8}\PYG{p}{)}\PYG{p}{,}\PYG{k}{INTENT}\PYG{p}{(}\PYG{n+nv}{IN}\PYG{p}{)} \PYG{k+kd}{::} \PYG{n+nv}{z}\PYG{p}{(}\PYG{n+nv}{nz}\PYG{p}{)} \PYG{c}{! (それ以外)}
\end{Verbatim}
\begin{quote}

シフト点.
\end{quote}

\begin{Verbatim}[commandchars=\\\{\}]
\PYG{k+kt}{INTEGER}\PYG{p}{,}\PYG{k}{INTENT}\PYG{p}{(}\PYG{n+nv}{IN}\PYG{p}{)} \PYG{k+kd}{::} \PYG{n+nv}{itermax}
\end{Verbatim}
\begin{quote}

リスタート用配列の割り付けのための最大反復回数.
これを \code{0} にした場合にはリスタート用配列を割りつけない
(したがって後述のリスタート用変数の出力を行えない)
\end{quote}

\begin{Verbatim}[commandchars=\\\{\}]
\PYG{k+kt}{REAL}\PYG{p}{(}\PYG{l+m+mi}{8}\PYG{p}{)}\PYG{p}{,}\PYG{k}{INTENT}\PYG{p}{(}\PYG{n+nv}{IN}\PYG{p}{)} \PYG{k+kd}{::} \PYG{n+nv}{threshold}
\end{Verbatim}
\begin{quote}

収束判定用しきい値.
シード方程式の残差ベクトルの2-ノルムがこの値を下回った時に収束したと判定する.
\end{quote}

\begin{Verbatim}[commandchars=\\\{\}]
\PYG{k+kt}{INTEGER}\PYG{p}{,}\PYG{k}{INTENT}\PYG{p}{(}\PYG{n+nv}{IN}\PYG{p}{)} \PYG{k+kd}{::} \PYG{n+nv}{comm}
\end{Verbatim}
\begin{quote}

MPI/Hybrid並列版のみ.
MPIのコミニュケーター( \code{MPI\_COMM\_WORLD} など)を入れる.
\end{quote}
\end{quote}


\subsection{*\_restart}
\label{komega_usage_ja:id4}\label{komega_usage_ja:restart}
リスタートを行う場合に {\hyperref[komega_usage_ja:init]{\emph{*\_init}}} の代わりに用いる.
ライブラリ内部変数の割り付けおよび初期化を行う.
シフト線形方程式を解く前に, 一番初めに実行する.

構文
\begin{quote}

Fortran (シリアル/OpenMP版)

\begin{Verbatim}[commandchars=\\\{\}]
\PYG{k}{CALL }\PYG{n+nv}{komega\PYGZus{}cg\PYGZus{}r\PYGZus{}restart}\PYG{p}{(}\PYG{n+nv}{ndim}\PYG{p}{,} \PYG{n+nv}{nl}\PYG{p}{,} \PYG{n+nv}{nz}\PYG{p}{,} \PYG{n+nv}{x}\PYG{p}{,} \PYG{n+nv}{z}\PYG{p}{,} \PYG{n+nv}{itermax}\PYG{p}{,} \PYG{n+nv}{threshold}\PYG{p}{,} \PYG{n+nv}{status}\PYG{p}{,} \PYG{p}{\PYGZam{}}
\PYG{p}{\PYGZam{}}                 \PYG{n+nv}{iter\PYGZus{}old}\PYG{p}{,} \PYG{n+nv}{v2}\PYG{p}{,} \PYG{n+nv}{v12}\PYG{p}{,} \PYG{n+nv}{alpha\PYGZus{}save}\PYG{p}{,} \PYG{n+nv}{beta\PYGZus{}save}\PYG{p}{,} \PYG{n+nv}{z\PYGZus{}seed}\PYG{p}{,} \PYG{n+nv}{r\PYGZus{}l\PYGZus{}save}\PYG{p}{)}
\PYG{k}{CALL }\PYG{n+nv}{komega\PYGZus{}cg\PYGZus{}c\PYGZus{}restart}\PYG{p}{(}\PYG{n+nv}{ndim}\PYG{p}{,} \PYG{n+nv}{nl}\PYG{p}{,} \PYG{n+nv}{nz}\PYG{p}{,} \PYG{n+nv}{x}\PYG{p}{,} \PYG{n+nv}{z}\PYG{p}{,} \PYG{n+nv}{itermax}\PYG{p}{,} \PYG{n+nv}{threshold}\PYG{p}{,} \PYG{n+nv}{status}\PYG{p}{,} \PYG{p}{\PYGZam{}}
\PYG{p}{\PYGZam{}}                 \PYG{n+nv}{iter\PYGZus{}old}\PYG{p}{,} \PYG{n+nv}{v2}\PYG{p}{,} \PYG{n+nv}{v12}\PYG{p}{,} \PYG{n+nv}{alpha\PYGZus{}save}\PYG{p}{,} \PYG{n+nv}{beta\PYGZus{}save}\PYG{p}{,} \PYG{n+nv}{z\PYGZus{}seed}\PYG{p}{,} \PYG{n+nv}{r\PYGZus{}l\PYGZus{}save}\PYG{p}{)}
\PYG{k}{CALL }\PYG{n+nv}{komega\PYGZus{}cocg\PYGZus{}restart}\PYG{p}{(}\PYG{n+nv}{ndim}\PYG{p}{,} \PYG{n+nv}{nl}\PYG{p}{,} \PYG{n+nv}{nz}\PYG{p}{,} \PYG{n+nv}{x}\PYG{p}{,} \PYG{n+nv}{z}\PYG{p}{,} \PYG{n+nv}{itermax}\PYG{p}{,} \PYG{n+nv}{threshold}\PYG{p}{,} \PYG{n+nv}{status}\PYG{p}{,} \PYG{p}{\PYGZam{}}
\PYG{p}{\PYGZam{}}                 \PYG{n+nv}{iter\PYGZus{}old}\PYG{p}{,} \PYG{n+nv}{v2}\PYG{p}{,} \PYG{n+nv}{v12}\PYG{p}{,} \PYG{n+nv}{alpha\PYGZus{}save}\PYG{p}{,} \PYG{n+nv}{beta\PYGZus{}save}\PYG{p}{,} \PYG{n+nv}{z\PYGZus{}seed}\PYG{p}{,} \PYG{n+nv}{r\PYGZus{}l\PYGZus{}save}\PYG{p}{)}
\PYG{k}{CALL }\PYG{n+nv}{komega\PYGZus{}bicg\PYGZus{}restart}\PYG{p}{(}\PYG{n+nv}{ndim}\PYG{p}{,} \PYG{n+nv}{nl}\PYG{p}{,} \PYG{n+nv}{nz}\PYG{p}{,} \PYG{n+nv}{x}\PYG{p}{,} \PYG{n+nv}{z}\PYG{p}{,} \PYG{n+nv}{itermax}\PYG{p}{,} \PYG{n+nv}{threshold}\PYG{p}{,} \PYG{n+nv}{status}\PYG{p}{,} \PYG{p}{\PYGZam{}}
\PYG{p}{\PYGZam{}}                 \PYG{n+nv}{iter\PYGZus{}old}\PYG{p}{,} \PYG{n+nv}{v2}\PYG{p}{,} \PYG{n+nv}{v12}\PYG{p}{,} \PYG{n+nv}{v4}\PYG{p}{,} \PYG{n+nv}{v14}\PYG{p}{,} \PYG{n+nv}{alpha\PYGZus{}save}\PYG{p}{,} \PYG{n+nv}{beta\PYGZus{}save}\PYG{p}{,} \PYG{p}{\PYGZam{}}
\PYG{p}{\PYGZam{}}                 \PYG{n+nv}{z\PYGZus{}seed}\PYG{p}{,} \PYG{n+nv}{r\PYGZus{}l\PYGZus{}save}\PYG{p}{)}
\end{Verbatim}

Fortran (MPI/ハイブリッド並列版)

\begin{Verbatim}[commandchars=\\\{\}]
\PYG{k}{CALL }\PYG{n+nv}{pkomega\PYGZus{}cg\PYGZus{}r\PYGZus{}restart}\PYG{p}{(}\PYG{n+nv}{ndim}\PYG{p}{,} \PYG{n+nv}{nl}\PYG{p}{,} \PYG{n+nv}{nz}\PYG{p}{,} \PYG{n+nv}{x}\PYG{p}{,} \PYG{n+nv}{z}\PYG{p}{,} \PYG{n+nv}{itermax}\PYG{p}{,} \PYG{n+nv}{threshold}\PYG{p}{,} \PYG{n+nv}{comm}\PYG{p}{,} \PYG{n+nv}{status}\PYG{p}{,} \PYG{p}{\PYGZam{}}
\PYG{p}{\PYGZam{}}                 \PYG{n+nv}{iter\PYGZus{}old}\PYG{p}{,} \PYG{n+nv}{v2}\PYG{p}{,} \PYG{n+nv}{v12}\PYG{p}{,} \PYG{n+nv}{alpha\PYGZus{}save}\PYG{p}{,} \PYG{n+nv}{beta\PYGZus{}save}\PYG{p}{,} \PYG{n+nv}{z\PYGZus{}seed}\PYG{p}{,} \PYG{n+nv}{r\PYGZus{}l\PYGZus{}save}\PYG{p}{)}
\PYG{k}{CALL }\PYG{n+nv}{pkomega\PYGZus{}cg\PYGZus{}c\PYGZus{}restart}\PYG{p}{(}\PYG{n+nv}{ndim}\PYG{p}{,} \PYG{n+nv}{nl}\PYG{p}{,} \PYG{n+nv}{nz}\PYG{p}{,} \PYG{n+nv}{x}\PYG{p}{,} \PYG{n+nv}{z}\PYG{p}{,} \PYG{n+nv}{itermax}\PYG{p}{,} \PYG{n+nv}{threshold}\PYG{p}{,} \PYG{n+nv}{comm}\PYG{p}{,} \PYG{n+nv}{status}\PYG{p}{,} \PYG{p}{\PYGZam{}}
\PYG{p}{\PYGZam{}}                 \PYG{n+nv}{iter\PYGZus{}old}\PYG{p}{,} \PYG{n+nv}{v2}\PYG{p}{,} \PYG{n+nv}{v12}\PYG{p}{,} \PYG{n+nv}{alpha\PYGZus{}save}\PYG{p}{,} \PYG{n+nv}{beta\PYGZus{}save}\PYG{p}{,} \PYG{n+nv}{z\PYGZus{}seed}\PYG{p}{,} \PYG{n+nv}{r\PYGZus{}l\PYGZus{}save}\PYG{p}{)}
\PYG{k}{CALL }\PYG{n+nv}{pkomega\PYGZus{}cocg\PYGZus{}restart}\PYG{p}{(}\PYG{n+nv}{ndim}\PYG{p}{,} \PYG{n+nv}{nl}\PYG{p}{,} \PYG{n+nv}{nz}\PYG{p}{,} \PYG{n+nv}{x}\PYG{p}{,} \PYG{n+nv}{z}\PYG{p}{,} \PYG{n+nv}{itermax}\PYG{p}{,} \PYG{n+nv}{threshold}\PYG{p}{,} \PYG{n+nv}{comm}\PYG{p}{,} \PYG{n+nv}{status}\PYG{p}{,} \PYG{p}{\PYGZam{}}
\PYG{p}{\PYGZam{}}                 \PYG{n+nv}{iter\PYGZus{}old}\PYG{p}{,} \PYG{n+nv}{v2}\PYG{p}{,} \PYG{n+nv}{v12}\PYG{p}{,} \PYG{n+nv}{alpha\PYGZus{}save}\PYG{p}{,} \PYG{n+nv}{beta\PYGZus{}save}\PYG{p}{,} \PYG{n+nv}{z\PYGZus{}seed}\PYG{p}{,} \PYG{n+nv}{r\PYGZus{}l\PYGZus{}save}\PYG{p}{)}
\PYG{k}{CALL }\PYG{n+nv}{pkomega\PYGZus{}bicg\PYGZus{}restart}\PYG{p}{(}\PYG{n+nv}{ndim}\PYG{p}{,} \PYG{n+nv}{nl}\PYG{p}{,} \PYG{n+nv}{nz}\PYG{p}{,} \PYG{n+nv}{x}\PYG{p}{,} \PYG{n+nv}{z}\PYG{p}{,} \PYG{n+nv}{itermax}\PYG{p}{,} \PYG{n+nv}{threshold}\PYG{p}{,} \PYG{n+nv}{comm}\PYG{p}{,} \PYG{n+nv}{status}\PYG{p}{,} \PYG{p}{\PYGZam{}}
\PYG{p}{\PYGZam{}}                 \PYG{n+nv}{iter\PYGZus{}old}\PYG{p}{,} \PYG{n+nv}{v2}\PYG{p}{,} \PYG{n+nv}{v12}\PYG{p}{,} \PYG{n+nv}{v4}\PYG{p}{,} \PYG{n+nv}{v14}\PYG{p}{,} \PYG{n+nv}{alpha\PYGZus{}save}\PYG{p}{,} \PYG{n+nv}{beta\PYGZus{}save}\PYG{p}{,} \PYG{p}{\PYGZam{}}
\PYG{p}{\PYGZam{}}                 \PYG{n+nv}{z\PYGZus{}seed}\PYG{p}{,} \PYG{n+nv}{r\PYGZus{}l\PYGZus{}save}\PYG{p}{)}
\end{Verbatim}

C/C++ (シリアル/OpenMP版)

\begin{Verbatim}[commandchars=\\\{\}]
\PYG{n}{komega\PYGZus{}cg\PYGZus{}r\PYGZus{}restart}\PYG{p}{(}\PYG{o}{\PYGZam{}}\PYG{n}{ndim}\PYG{p}{,} \PYG{o}{\PYGZam{}}\PYG{n}{nl}\PYG{p}{,} \PYG{o}{\PYGZam{}}\PYG{n}{nz}\PYG{p}{,} \PYG{n}{x}\PYG{p}{,} \PYG{n}{z}\PYG{p}{,} \PYG{o}{\PYGZam{}}\PYG{n}{itermax}\PYG{p}{,} \PYG{o}{\PYGZam{}}\PYG{n}{threshold}\PYG{p}{,} \PYG{n}{status}\PYG{p}{,} \PYG{o}{\PYGZam{}}
\PYG{o}{\PYGZam{}}                 \PYG{o}{\PYGZam{}}\PYG{n}{iter\PYGZus{}old}\PYG{p}{,} \PYG{n}{v2}\PYG{p}{,} \PYG{n}{v12}\PYG{p}{,} \PYG{n}{alpha\PYGZus{}save}\PYG{p}{,} \PYG{n}{beta\PYGZus{}save}\PYG{p}{,} \PYG{o}{\PYGZam{}}\PYG{n}{z\PYGZus{}seed}\PYG{p}{,} \PYG{n}{r\PYGZus{}l\PYGZus{}save}\PYG{p}{)}\PYG{p}{;}
\PYG{n}{komega\PYGZus{}cg\PYGZus{}c\PYGZus{}restart}\PYG{p}{(}\PYG{o}{\PYGZam{}}\PYG{n}{ndim}\PYG{p}{,} \PYG{o}{\PYGZam{}}\PYG{n}{nl}\PYG{p}{,} \PYG{o}{\PYGZam{}}\PYG{n}{nz}\PYG{p}{,} \PYG{n}{x}\PYG{p}{,} \PYG{n}{z}\PYG{p}{,} \PYG{o}{\PYGZam{}}\PYG{n}{itermax}\PYG{p}{,} \PYG{o}{\PYGZam{}}\PYG{n}{threshold}\PYG{p}{,} \PYG{n}{status}\PYG{p}{,} \PYG{o}{\PYGZam{}}
\PYG{o}{\PYGZam{}}                 \PYG{o}{\PYGZam{}}\PYG{n}{iter\PYGZus{}old}\PYG{p}{,} \PYG{n}{v2}\PYG{p}{,} \PYG{n}{v12}\PYG{p}{,} \PYG{n}{alpha\PYGZus{}save}\PYG{p}{,} \PYG{n}{beta\PYGZus{}save}\PYG{p}{,} \PYG{o}{\PYGZam{}}\PYG{n}{z\PYGZus{}seed}\PYG{p}{,} \PYG{n}{r\PYGZus{}l\PYGZus{}save}\PYG{p}{)}\PYG{p}{;}
\PYG{n}{komega\PYGZus{}cocg\PYGZus{}restart}\PYG{p}{(}\PYG{o}{\PYGZam{}}\PYG{n}{ndim}\PYG{p}{,} \PYG{o}{\PYGZam{}}\PYG{n}{nl}\PYG{p}{,} \PYG{o}{\PYGZam{}}\PYG{n}{nz}\PYG{p}{,} \PYG{n}{x}\PYG{p}{,} \PYG{n}{z}\PYG{p}{,} \PYG{o}{\PYGZam{}}\PYG{n}{itermax}\PYG{p}{,} \PYG{o}{\PYGZam{}}\PYG{n}{threshold}\PYG{p}{,} \PYG{n}{status}\PYG{p}{,} \PYG{o}{\PYGZam{}}
\PYG{o}{\PYGZam{}}                 \PYG{o}{\PYGZam{}}\PYG{n}{iter\PYGZus{}old}\PYG{p}{,} \PYG{n}{v2}\PYG{p}{,} \PYG{n}{v12}\PYG{p}{,} \PYG{n}{alpha\PYGZus{}save}\PYG{p}{,} \PYG{n}{beta\PYGZus{}save}\PYG{p}{,} \PYG{o}{\PYGZam{}}\PYG{n}{z\PYGZus{}seed}\PYG{p}{,} \PYG{n}{r\PYGZus{}l\PYGZus{}save}\PYG{p}{)}\PYG{p}{;}
\PYG{n}{komega\PYGZus{}bicg\PYGZus{}restart}\PYG{p}{(}\PYG{o}{\PYGZam{}}\PYG{n}{ndim}\PYG{p}{,} \PYG{o}{\PYGZam{}}\PYG{n}{nl}\PYG{p}{,} \PYG{o}{\PYGZam{}}\PYG{n}{nz}\PYG{p}{,} \PYG{n}{x}\PYG{p}{,} \PYG{n}{z}\PYG{p}{,} \PYG{o}{\PYGZam{}}\PYG{n}{itermax}\PYG{p}{,} \PYG{o}{\PYGZam{}}\PYG{n}{threshold}\PYG{p}{,} \PYG{n}{status}\PYG{p}{,} \PYG{o}{\PYGZam{}}
\PYG{o}{\PYGZam{}}                 \PYG{o}{\PYGZam{}}\PYG{n}{iter\PYGZus{}old}\PYG{p}{,} \PYG{n}{v2}\PYG{p}{,} \PYG{n}{v12}\PYG{p}{,} \PYG{n}{v4}\PYG{p}{,} \PYG{n}{v14}\PYG{p}{,} \PYG{n}{alpha\PYGZus{}save}\PYG{p}{,} \PYG{n}{beta\PYGZus{}save}\PYG{p}{,} \PYG{o}{\PYGZam{}}
\PYG{o}{\PYGZam{}}                 \PYG{o}{\PYGZam{}}\PYG{n}{z\PYGZus{}seed}\PYG{p}{,} \PYG{n}{r\PYGZus{}l\PYGZus{}save}\PYG{p}{)}\PYG{p}{;}
\end{Verbatim}

C/C++ (MPI/ハイブリッド並列版)

\begin{Verbatim}[commandchars=\\\{\}]
\PYG{n}{pkomega\PYGZus{}cg\PYGZus{}r\PYGZus{}restart}\PYG{p}{(}\PYG{o}{\PYGZam{}}\PYG{n}{ndim}\PYG{p}{,} \PYG{o}{\PYGZam{}}\PYG{n}{nl}\PYG{p}{,} \PYG{o}{\PYGZam{}}\PYG{n}{nz}\PYG{p}{,} \PYG{n}{x}\PYG{p}{,} \PYG{n}{z}\PYG{p}{,} \PYG{o}{\PYGZam{}}\PYG{n}{itermax}\PYG{p}{,} \PYG{o}{\PYGZam{}}\PYG{n}{threshold}\PYG{p}{,} \PYG{o}{\PYGZam{}}\PYG{n}{comm}\PYG{p}{,} \PYG{n}{status}\PYG{p}{,} \PYG{o}{\PYGZam{}}
\PYG{o}{\PYGZam{}}                 \PYG{o}{\PYGZam{}}\PYG{n}{iter\PYGZus{}old}\PYG{p}{,} \PYG{n}{v2}\PYG{p}{,} \PYG{n}{v12}\PYG{p}{,} \PYG{n}{alpha\PYGZus{}save}\PYG{p}{,} \PYG{n}{beta\PYGZus{}save}\PYG{p}{,} \PYG{o}{\PYGZam{}}\PYG{n}{z\PYGZus{}seed}\PYG{p}{,} \PYG{n}{r\PYGZus{}l\PYGZus{}save}\PYG{p}{)}\PYG{p}{;}
\PYG{n}{pkomega\PYGZus{}cg\PYGZus{}c\PYGZus{}restart}\PYG{p}{(}\PYG{o}{\PYGZam{}}\PYG{n}{ndim}\PYG{p}{,} \PYG{o}{\PYGZam{}}\PYG{n}{nl}\PYG{p}{,} \PYG{o}{\PYGZam{}}\PYG{n}{nz}\PYG{p}{,} \PYG{n}{x}\PYG{p}{,} \PYG{n}{z}\PYG{p}{,} \PYG{o}{\PYGZam{}}\PYG{n}{itermax}\PYG{p}{,} \PYG{o}{\PYGZam{}}\PYG{n}{threshold}\PYG{p}{,} \PYG{o}{\PYGZam{}}\PYG{n}{comm}\PYG{p}{,} \PYG{n}{status}\PYG{p}{,} \PYG{o}{\PYGZam{}}
\PYG{o}{\PYGZam{}}                 \PYG{o}{\PYGZam{}}\PYG{n}{iter\PYGZus{}old}\PYG{p}{,} \PYG{n}{v2}\PYG{p}{,} \PYG{n}{v12}\PYG{p}{,} \PYG{n}{alpha\PYGZus{}save}\PYG{p}{,} \PYG{n}{beta\PYGZus{}save}\PYG{p}{,} \PYG{o}{\PYGZam{}}\PYG{n}{z\PYGZus{}seed}\PYG{p}{,} \PYG{n}{r\PYGZus{}l\PYGZus{}save}\PYG{p}{)}\PYG{p}{;}
\PYG{n}{pkomega\PYGZus{}cocg\PYGZus{}restart}\PYG{p}{(}\PYG{o}{\PYGZam{}}\PYG{n}{ndim}\PYG{p}{,} \PYG{o}{\PYGZam{}}\PYG{n}{nl}\PYG{p}{,} \PYG{o}{\PYGZam{}}\PYG{n}{nz}\PYG{p}{,} \PYG{n}{x}\PYG{p}{,} \PYG{n}{z}\PYG{p}{,} \PYG{o}{\PYGZam{}}\PYG{n}{itermax}\PYG{p}{,} \PYG{o}{\PYGZam{}}\PYG{n}{threshold}\PYG{p}{,} \PYG{o}{\PYGZam{}}\PYG{n}{comm}\PYG{p}{,} \PYG{n}{status}\PYG{p}{,} \PYG{o}{\PYGZam{}}
\PYG{o}{\PYGZam{}}                 \PYG{o}{\PYGZam{}}\PYG{n}{iter\PYGZus{}old}\PYG{p}{,} \PYG{n}{v2}\PYG{p}{,} \PYG{n}{v12}\PYG{p}{,} \PYG{n}{alpha\PYGZus{}save}\PYG{p}{,} \PYG{n}{beta\PYGZus{}save}\PYG{p}{,} \PYG{o}{\PYGZam{}}\PYG{n}{z\PYGZus{}seed}\PYG{p}{,} \PYG{n}{r\PYGZus{}l\PYGZus{}save}\PYG{p}{)}\PYG{p}{;}
\PYG{n}{pkomega\PYGZus{}bicg\PYGZus{}restart}\PYG{p}{(}\PYG{o}{\PYGZam{}}\PYG{n}{ndim}\PYG{p}{,} \PYG{o}{\PYGZam{}}\PYG{n}{nl}\PYG{p}{,} \PYG{o}{\PYGZam{}}\PYG{n}{nz}\PYG{p}{,} \PYG{n}{x}\PYG{p}{,} \PYG{n}{z}\PYG{p}{,} \PYG{o}{\PYGZam{}}\PYG{n}{itermax}\PYG{p}{,} \PYG{o}{\PYGZam{}}\PYG{n}{threshold}\PYG{p}{,} \PYG{o}{\PYGZam{}}\PYG{n}{comm}\PYG{p}{,} \PYG{n}{status}\PYG{p}{,} \PYG{o}{\PYGZam{}}
\PYG{o}{\PYGZam{}}                 \PYG{o}{\PYGZam{}}\PYG{n}{iter\PYGZus{}old}\PYG{p}{,} \PYG{n}{v2}\PYG{p}{,} \PYG{n}{v12}\PYG{p}{,} \PYG{n}{v4}\PYG{p}{,} \PYG{n}{v14}\PYG{p}{,} \PYG{n}{alpha\PYGZus{}save}\PYG{p}{,} \PYG{n}{beta\PYGZus{}save}\PYG{p}{,} \PYG{o}{\PYGZam{}}
\PYG{o}{\PYGZam{}}                 \PYG{o}{\PYGZam{}}\PYG{n}{z\PYGZus{}seed}\PYG{p}{,} \PYG{n}{r\PYGZus{}l\PYGZus{}save}\PYG{p}{)}\PYG{p}{;}
\end{Verbatim}
\end{quote}

パラメーター
\begin{quote}

\begin{Verbatim}[commandchars=\\\{\}]
\PYG{k+kt}{INTEGER}\PYG{p}{,}\PYG{k}{INTENT}\PYG{p}{(}\PYG{n+nv}{IN}\PYG{p}{)} \PYG{k+kd}{::} \PYG{n+nv}{ndim}
\PYG{k+kt}{INTEGER}\PYG{p}{,}\PYG{k}{INTENT}\PYG{p}{(}\PYG{n+nv}{IN}\PYG{p}{)} \PYG{k+kd}{::} \PYG{n+nv}{nl}
\PYG{k+kt}{INTEGER}\PYG{p}{,}\PYG{k}{INTENT}\PYG{p}{(}\PYG{n+nv}{IN}\PYG{p}{)} \PYG{k+kd}{::} \PYG{n+nv}{nz}
\PYG{k+kt}{REAL}\PYG{p}{(}\PYG{l+m+mi}{8}\PYG{p}{)}\PYG{p}{,}\PYG{k}{INTENT}\PYG{p}{(}\PYG{n+nv}{OUT}\PYG{p}{)} \PYG{k+kd}{::} \PYG{n+nv}{x}\PYG{p}{(}\PYG{n+nv}{nl}\PYG{o}{*}\PYG{n+nv}{nz}\PYG{p}{)}
\PYG{k+kt}{REAL}\PYG{p}{(}\PYG{l+m+mi}{8}\PYG{p}{)}\PYG{p}{,}\PYG{k}{INTENT}\PYG{p}{(}\PYG{n+nv}{IN}\PYG{p}{)} \PYG{k+kd}{::} \PYG{n+nv}{z}\PYG{p}{(}\PYG{n+nv}{nz}\PYG{p}{)} \PYG{c}{! (\PYGZdq{}CG\PYGZus{}R\PYGZus{}restart\PYGZdq{}, \PYGZdq{}cg\PYGZus{}c\PYGZus{}restart\PYGZdq{} の場合)}
\PYG{k+kt}{COMPLEX}\PYG{p}{(}\PYG{l+m+mi}{8}\PYG{p}{)}\PYG{p}{,}\PYG{k}{INTENT}\PYG{p}{(}\PYG{n+nv}{IN}\PYG{p}{)} \PYG{k+kd}{::} \PYG{n+nv}{z}\PYG{p}{(}\PYG{n+nv}{nz}\PYG{p}{)} \PYG{c}{! (それ以外)}
\PYG{k+kt}{INTEGER}\PYG{p}{,}\PYG{k}{INTENT}\PYG{p}{(}\PYG{n+nv}{IN}\PYG{p}{)} \PYG{k+kd}{::} \PYG{n+nv}{itermax}
\PYG{k+kt}{REAL}\PYG{p}{(}\PYG{l+m+mi}{8}\PYG{p}{)}\PYG{p}{,}\PYG{k}{INTENT}\PYG{p}{(}\PYG{n+nv}{IN}\PYG{p}{)} \PYG{k+kd}{::} \PYG{n+nv}{threshold}
\PYG{k+kt}{INTEGER}\PYG{p}{,}\PYG{k}{INTENT}\PYG{p}{(}\PYG{n+nv}{IN}\PYG{p}{)} \PYG{k+kd}{::} \PYG{n+nv}{comm}
\end{Verbatim}
\begin{quote}

{\hyperref[komega_usage_ja:init]{\emph{*\_init}}} と同様.
\end{quote}

\begin{Verbatim}[commandchars=\\\{\}]
\PYG{k+kt}{INTEGER}\PYG{p}{,}\PYG{k}{INTENT}\PYG{p}{(}\PYG{n+nv}{OUT}\PYG{p}{)} \PYG{k+kd}{::} \PYG{n+nv}{status}\PYG{p}{(}\PYG{l+m+mi}{3}\PYG{p}{)}
\end{Verbatim}
\begin{quote}

エラーコードを返す.

第一成分( \code{status(1)})
\begin{quote}

解が収束した場合,
もしくは計算が破綻した場合には現在の総反復回数に
マイナスが付いた値が返される.
それ以外の場合には現在の総反復回数(マイナスが付かない)が返される.
\code{status(1)} が正の値の時のみ反復を続行できる.
それ以外の場合は反復を進めても有意な結果は得られない.
\end{quote}

第二成分( \code{status(2)})
\begin{quote}

\code{itermax} を有限にして, かつ \code{itermax} 回の反復で
収束に達しなかった場合には \code{1} が返される.
\(\alpha\) が発散した場合には \code{2} が返される.
\(\pi_{\rm seed}\) が0にになった場合には \code{3} が返される.
\code{COCG\_restart} もしくは \code{BiCG\_restart} で,
残差ベクトルと影の残差ベクトルが直交した場合には \code{4} が返される.
それ以外の場合には \code{0} が返される.
\end{quote}

第三成分( \code{status(3)})
\begin{quote}

シード点のindexが返される.
\end{quote}
\end{quote}

\begin{Verbatim}[commandchars=\\\{\}]
\PYG{k+kt}{INTEGER}\PYG{p}{,}\PYG{k}{INTENT}\PYG{p}{(}\PYG{n+nv}{IN}\PYG{p}{)} \PYG{k+kd}{::} \PYG{n+nv}{iter\PYGZus{}old}
\end{Verbatim}
\begin{quote}

先行する計算での反復回数.
\end{quote}

\begin{Verbatim}[commandchars=\\\{\}]
\PYG{k+kt}{REAL}\PYG{p}{(}\PYG{l+m+mi}{8}\PYG{p}{)}\PYG{p}{,}\PYG{k}{INTENT}\PYG{p}{(}\PYG{n+nv}{IN}\PYG{p}{)} \PYG{k+kd}{::} \PYG{n+nv}{v2}\PYG{p}{(}\PYG{n+nv}{ndim}\PYG{p}{)} \PYG{c}{! (\PYGZdq{}CG\PYGZus{}R\PYGZus{}restart\PYGZdq{} の場合)}
\PYG{k+kt}{COMPLEX}\PYG{p}{(}\PYG{l+m+mi}{8}\PYG{p}{)}\PYG{p}{,}\PYG{k}{INTENT}\PYG{p}{(}\PYG{n+nv}{IN}\PYG{p}{)} \PYG{k+kd}{::} \PYG{n+nv}{v2}\PYG{p}{(}\PYG{n+nv}{ndim}\PYG{p}{)} \PYG{c}{! (それ以外)}
\end{Verbatim}
\begin{quote}

先行する計算での最後の残差ベクトル.
\end{quote}

\begin{Verbatim}[commandchars=\\\{\}]
\PYG{k+kt}{REAL}\PYG{p}{(}\PYG{l+m+mi}{8}\PYG{p}{)}\PYG{p}{,}\PYG{k}{INTENT}\PYG{p}{(}\PYG{n+nv}{IN}\PYG{p}{)} \PYG{k+kd}{::} \PYG{n+nv}{v12}\PYG{p}{(}\PYG{n+nv}{ndim}\PYG{p}{)} \PYG{c}{! (\PYGZdq{}CG\PYGZus{}R\PYGZus{}restart\PYGZdq{} の場合)}
\PYG{k+kt}{COMPLEX}\PYG{p}{(}\PYG{l+m+mi}{8}\PYG{p}{)}\PYG{p}{,}\PYG{k}{INTENT}\PYG{p}{(}\PYG{n+nv}{IN}\PYG{p}{)} \PYG{k+kd}{::} \PYG{n+nv}{v12}\PYG{p}{(}\PYG{n+nv}{ndim}\PYG{p}{)} \PYG{c}{! (それ以外)}
\end{Verbatim}
\begin{quote}

先行する計算での最後から2番目の残差ベクトル.
\end{quote}

\begin{Verbatim}[commandchars=\\\{\}]
\PYG{k+kt}{REAL}\PYG{p}{(}\PYG{l+m+mi}{8}\PYG{p}{)}\PYG{p}{,}\PYG{k}{INTENT}\PYG{p}{(}\PYG{n+nv}{IN}\PYG{p}{)} \PYG{k+kd}{::} \PYG{n+nv}{alpha\PYGZus{}save}\PYG{p}{(}\PYG{n+nv}{iter\PYGZus{}old}\PYG{p}{)} \PYG{c}{! (\PYGZdq{}CG\PYGZus{}R\PYGZus{}restart\PYGZdq{}, \PYGZdq{}cg\PYGZus{}c\PYGZus{}restart\PYGZdq{}の場合)}
\PYG{k+kt}{COMPLEX}\PYG{p}{(}\PYG{l+m+mi}{8}\PYG{p}{)}\PYG{p}{,}\PYG{k}{INTENT}\PYG{p}{(}\PYG{n+nv}{IN}\PYG{p}{)} \PYG{k+kd}{::} \PYG{n+nv}{alpha\PYGZus{}save}\PYG{p}{(}\PYG{n+nv}{iter\PYGZus{}old}\PYG{p}{)} \PYG{c}{! (それ以外)}
\end{Verbatim}
\begin{quote}

先行する計算での各反復での(Bi)CG法のパラメーター \(\alpha\).
\end{quote}

\begin{Verbatim}[commandchars=\\\{\}]
\PYG{k+kt}{REAL}\PYG{p}{(}\PYG{l+m+mi}{8}\PYG{p}{)}\PYG{p}{,}\PYG{k}{INTENT}\PYG{p}{(}\PYG{n+nv}{IN}\PYG{p}{)} \PYG{k+kd}{::} \PYG{n+nv}{beta\PYGZus{}save}\PYG{p}{(}\PYG{n+nv}{iter\PYGZus{}old}\PYG{p}{)} \PYG{c}{! (\PYGZdq{}CG\PYGZus{}R\PYGZus{}restart\PYGZdq{}, \PYGZdq{}cg\PYGZus{}c\PYGZus{}restart\PYGZdq{}の場合)}
\PYG{k+kt}{COMPLEX}\PYG{p}{(}\PYG{l+m+mi}{8}\PYG{p}{)}\PYG{p}{,}\PYG{k}{INTENT}\PYG{p}{(}\PYG{n+nv}{IN}\PYG{p}{)} \PYG{k+kd}{::} \PYG{n+nv}{beta\PYGZus{}save}\PYG{p}{(}\PYG{n+nv}{iter\PYGZus{}old}\PYG{p}{)} \PYG{c}{! (それ以外)}
\end{Verbatim}
\begin{quote}

先行する計算での各反復での(Bi)CG法のパラメーター \(\beta\).
\end{quote}

\begin{Verbatim}[commandchars=\\\{\}]
\PYG{k+kt}{REAL}\PYG{p}{(}\PYG{l+m+mi}{8}\PYG{p}{)}\PYG{p}{,}\PYG{k}{INTENT}\PYG{p}{(}\PYG{n+nv}{IN}\PYG{p}{)} \PYG{k+kd}{::} \PYG{n+nv}{z\PYGZus{}seed} \PYG{c}{! (\PYGZdq{}CG\PYGZus{}R\PYGZus{}restart\PYGZdq{}, \PYGZdq{}cg\PYGZus{}c\PYGZus{}restart\PYGZdq{}の場合)}
\PYG{k+kt}{COMPLEX}\PYG{p}{(}\PYG{l+m+mi}{8}\PYG{p}{)}\PYG{p}{,}\PYG{k}{INTENT}\PYG{p}{(}\PYG{n+nv}{IN}\PYG{p}{)} \PYG{k+kd}{::} \PYG{n+nv}{z\PYGZus{}seed} \PYG{c}{! (それ以外)}
\end{Verbatim}
\begin{quote}

先行する計算でのシードシフト.
\end{quote}

\begin{Verbatim}[commandchars=\\\{\}]
\PYG{k+kt}{REAL}\PYG{p}{(}\PYG{l+m+mi}{8}\PYG{p}{)}\PYG{p}{,}\PYG{k}{INTENT}\PYG{p}{(}\PYG{n+nv}{IN}\PYG{p}{)} \PYG{k+kd}{::} \PYG{n+nv}{r\PYGZus{}l\PYGZus{}save}\PYG{p}{(}\PYG{n+nv}{nl}\PYG{p}{,}\PYG{n+nv}{iter\PYGZus{}old}\PYG{p}{)} \PYG{c}{! (\PYGZdq{}CG\PYGZus{}R\PYGZus{}restart\PYGZdq{}の場合)}
\PYG{k+kt}{COMPLEX}\PYG{p}{(}\PYG{l+m+mi}{8}\PYG{p}{)}\PYG{p}{,}\PYG{k}{INTENT}\PYG{p}{(}\PYG{n+nv}{IN}\PYG{p}{)} \PYG{k+kd}{::} \PYG{n+nv}{r\PYGZus{}l\PYGZus{}save}\PYG{p}{(}\PYG{n+nv}{nl}\PYG{p}{,}\PYG{n+nv}{iter\PYGZus{}old}\PYG{p}{)} \PYG{c}{! (それ以外)}
\end{Verbatim}
\begin{quote}

先行する計算での各反復での射影された残差ベクトル.
\end{quote}

\begin{Verbatim}[commandchars=\\\{\}]
\PYG{k+kt}{REAL}\PYG{p}{(}\PYG{l+m+mi}{8}\PYG{p}{)}\PYG{p}{,}\PYG{k}{INTENT}\PYG{p}{(}\PYG{n+nv}{IN}\PYG{p}{)} \PYG{k+kd}{::} \PYG{n+nv}{v4}\PYG{p}{(}\PYG{n+nv}{ndim}\PYG{p}{)} \PYG{c}{! (\PYGZdq{}CG\PYGZus{}R\PYGZus{}restart\PYGZdq{} の場合)}
\PYG{k+kt}{COMPLEX}\PYG{p}{(}\PYG{l+m+mi}{8}\PYG{p}{)}\PYG{p}{,}\PYG{k}{INTENT}\PYG{p}{(}\PYG{n+nv}{IN}\PYG{p}{)} \PYG{k+kd}{::} \PYG{n+nv}{v4}\PYG{p}{(}\PYG{n+nv}{ndim}\PYG{p}{)} \PYG{c}{! (それ以外)}
\end{Verbatim}
\begin{quote}

\code{BiCG\_restart} の場合のみ使用.
先行する計算での最後の影の残差ベクトル.
\end{quote}

\begin{Verbatim}[commandchars=\\\{\}]
\PYG{k+kt}{REAL}\PYG{p}{(}\PYG{l+m+mi}{8}\PYG{p}{)}\PYG{p}{,}\PYG{k}{INTENT}\PYG{p}{(}\PYG{n+nv}{IN}\PYG{p}{)} \PYG{k+kd}{::} \PYG{n+nv}{v14}\PYG{p}{(}\PYG{n+nv}{ndim}\PYG{p}{)} \PYG{c}{! (\PYGZdq{}CG\PYGZus{}R\PYGZus{}restart\PYGZdq{} の場合)}
\PYG{k+kt}{COMPLEX}\PYG{p}{(}\PYG{l+m+mi}{8}\PYG{p}{)}\PYG{p}{,}\PYG{k}{INTENT}\PYG{p}{(}\PYG{n+nv}{IN}\PYG{p}{)} \PYG{k+kd}{::} \PYG{n+nv}{v14}\PYG{p}{(}\PYG{n+nv}{ndim}\PYG{p}{)} \PYG{c}{! (それ以外)}
\end{Verbatim}
\begin{quote}

\code{BiCG\_restart} の場合のみ使用.
先行する計算での最後から2番目の影の残差ベクトル.
\end{quote}
\end{quote}


\subsection{*\_update}
\label{komega_usage_ja:id5}\label{komega_usage_ja:update}
ループ内で行列ベクトル積と交互に呼ばれて解を更新する.

構文
\begin{quote}

Fortran (シリアル/OpenMPI版)

\begin{Verbatim}[commandchars=\\\{\}]
\PYG{k}{CALL }\PYG{n+nv}{komega\PYGZus{}cg\PYGZus{}r\PYGZus{}update}\PYG{p}{(}\PYG{n+nv}{v12}\PYG{p}{,} \PYG{n+nv}{v2}\PYG{p}{,} \PYG{n+nv}{x}\PYG{p}{,} \PYG{n+nv}{r\PYGZus{}l}\PYG{p}{,} \PYG{n+nv}{status}\PYG{p}{)}
\PYG{k}{CALL }\PYG{n+nv}{komega\PYGZus{}cg\PYGZus{}c\PYGZus{}update}\PYG{p}{(}\PYG{n+nv}{v12}\PYG{p}{,} \PYG{n+nv}{v2}\PYG{p}{,} \PYG{n+nv}{x}\PYG{p}{,} \PYG{n+nv}{r\PYGZus{}l}\PYG{p}{,} \PYG{n+nv}{status}\PYG{p}{)}
\PYG{k}{CALL }\PYG{n+nv}{komega\PYGZus{}cocg\PYGZus{}update}\PYG{p}{(}\PYG{n+nv}{v12}\PYG{p}{,} \PYG{n+nv}{v2}\PYG{p}{,} \PYG{n+nv}{x}\PYG{p}{,} \PYG{n+nv}{r\PYGZus{}l}\PYG{p}{,} \PYG{n+nv}{status}\PYG{p}{)}
\PYG{k}{CALL }\PYG{n+nv}{komega\PYGZus{}bicg\PYGZus{}update}\PYG{p}{(}\PYG{n+nv}{v12}\PYG{p}{,} \PYG{n+nv}{v2}\PYG{p}{,} \PYG{n+nv}{v14}\PYG{p}{,} \PYG{n+nv}{v4}\PYG{p}{,} \PYG{n+nv}{x}\PYG{p}{,} \PYG{n+nv}{r\PYGZus{}l}\PYG{p}{,} \PYG{n+nv}{status}\PYG{p}{)}
\end{Verbatim}

Fortran (MPI/ハイブリッド並列版)

\begin{Verbatim}[commandchars=\\\{\}]
\PYG{k}{CALL }\PYG{n+nv}{pkomega\PYGZus{}cg\PYGZus{}r\PYGZus{}update}\PYG{p}{(}\PYG{n+nv}{v12}\PYG{p}{,} \PYG{n+nv}{v2}\PYG{p}{,} \PYG{n+nv}{x}\PYG{p}{,} \PYG{n+nv}{r\PYGZus{}l}\PYG{p}{,} \PYG{n+nv}{status}\PYG{p}{)}
\PYG{k}{CALL }\PYG{n+nv}{pkomega\PYGZus{}cg\PYGZus{}c\PYGZus{}update}\PYG{p}{(}\PYG{n+nv}{v12}\PYG{p}{,} \PYG{n+nv}{v2}\PYG{p}{,} \PYG{n+nv}{x}\PYG{p}{,} \PYG{n+nv}{r\PYGZus{}l}\PYG{p}{,} \PYG{n+nv}{status}\PYG{p}{)}
\PYG{k}{CALL }\PYG{n+nv}{pkomega\PYGZus{}cocg\PYGZus{}update}\PYG{p}{(}\PYG{n+nv}{v12}\PYG{p}{,} \PYG{n+nv}{v2}\PYG{p}{,} \PYG{n+nv}{x}\PYG{p}{,} \PYG{n+nv}{r\PYGZus{}l}\PYG{p}{,} \PYG{n+nv}{status}\PYG{p}{)}
\PYG{k}{CALL }\PYG{n+nv}{pkomega\PYGZus{}bicg\PYGZus{}update}\PYG{p}{(}\PYG{n+nv}{v12}\PYG{p}{,} \PYG{n+nv}{v2}\PYG{p}{,} \PYG{n+nv}{v14}\PYG{p}{,} \PYG{n+nv}{v4}\PYG{p}{,} \PYG{n+nv}{x}\PYG{p}{,} \PYG{n+nv}{r\PYGZus{}l}\PYG{p}{,} \PYG{n+nv}{status}\PYG{p}{)}
\end{Verbatim}

C/C++ (シリアル/OpenMPI版)

\begin{Verbatim}[commandchars=\\\{\}]
\PYG{n}{komega\PYGZus{}cg\PYGZus{}r\PYGZus{}update}\PYG{p}{(}\PYG{n}{v12}\PYG{p}{,} \PYG{n}{v2}\PYG{p}{,} \PYG{n}{x}\PYG{p}{,} \PYG{n}{r\PYGZus{}l}\PYG{p}{,} \PYG{n}{status}\PYG{p}{)}\PYG{p}{;}
\PYG{n}{komega\PYGZus{}cg\PYGZus{}c\PYGZus{}update}\PYG{p}{(}\PYG{n}{v12}\PYG{p}{,} \PYG{n}{v2}\PYG{p}{,} \PYG{n}{x}\PYG{p}{,} \PYG{n}{r\PYGZus{}l}\PYG{p}{,} \PYG{n}{status}\PYG{p}{)}\PYG{p}{;}
\PYG{n}{komega\PYGZus{}cocg\PYGZus{}update}\PYG{p}{(}\PYG{n}{v12}\PYG{p}{,} \PYG{n}{v2}\PYG{p}{,} \PYG{n}{x}\PYG{p}{,} \PYG{n}{r\PYGZus{}l}\PYG{p}{,} \PYG{n}{status}\PYG{p}{)}\PYG{p}{;}
\PYG{n}{komega\PYGZus{}bicg\PYGZus{}update}\PYG{p}{(}\PYG{n}{v12}\PYG{p}{,} \PYG{n}{v2}\PYG{p}{,} \PYG{n}{v14}\PYG{p}{,} \PYG{n}{v4}\PYG{p}{,} \PYG{n}{x}\PYG{p}{,} \PYG{n}{r\PYGZus{}l}\PYG{p}{,} \PYG{n}{status}\PYG{p}{)}\PYG{p}{;}
\end{Verbatim}

C/C++ (MPI/ハイブリッド並列版)

\begin{Verbatim}[commandchars=\\\{\}]
\PYG{n}{pkomega\PYGZus{}cg\PYGZus{}r\PYGZus{}update}\PYG{p}{(}\PYG{n}{v12}\PYG{p}{,} \PYG{n}{v2}\PYG{p}{,} \PYG{n}{x}\PYG{p}{,} \PYG{n}{r\PYGZus{}l}\PYG{p}{,} \PYG{n}{status}\PYG{p}{)}\PYG{p}{;}
\PYG{n}{pkomega\PYGZus{}cg\PYGZus{}c\PYGZus{}update}\PYG{p}{(}\PYG{n}{v12}\PYG{p}{,} \PYG{n}{v2}\PYG{p}{,} \PYG{n}{x}\PYG{p}{,} \PYG{n}{r\PYGZus{}l}\PYG{p}{,} \PYG{n}{status}\PYG{p}{)}\PYG{p}{;}
\PYG{n}{pkomega\PYGZus{}cocg\PYGZus{}update}\PYG{p}{(}\PYG{n}{v12}\PYG{p}{,} \PYG{n}{v2}\PYG{p}{,} \PYG{n}{x}\PYG{p}{,} \PYG{n}{r\PYGZus{}l}\PYG{p}{,} \PYG{n}{status}\PYG{p}{)}\PYG{p}{;}
\PYG{n}{pkomega\PYGZus{}bicg\PYGZus{}update}\PYG{p}{(}\PYG{n}{v12}\PYG{p}{,} \PYG{n}{v2}\PYG{p}{,} \PYG{n}{v14}\PYG{p}{,} \PYG{n}{v4}\PYG{p}{,} \PYG{n}{x}\PYG{p}{,} \PYG{n}{r\PYGZus{}l}\PYG{p}{,} \PYG{n}{status}\PYG{p}{)}\PYG{p}{;}
\end{Verbatim}

パラメーター

\begin{Verbatim}[commandchars=\\\{\}]
\PYG{k+kt}{REAL}\PYG{p}{(}\PYG{l+m+mi}{8}\PYG{p}{)}\PYG{p}{,}\PYG{k}{INTENT}\PYG{p}{(}\PYG{n+nv}{INOUT}\PYG{p}{)} \PYG{k+kd}{::} \PYG{n+nv}{v12}\PYG{p}{(}\PYG{n+nv}{ndim}\PYG{p}{)} \PYG{c}{! (\PYGZdq{}CG\PYGZus{}R\PYGZus{}update\PYGZdq{} の場合)}
\PYG{k+kt}{COMPLEX}\PYG{p}{(}\PYG{l+m+mi}{8}\PYG{p}{)}\PYG{p}{,}\PYG{k}{INTENT}\PYG{p}{(}\PYG{n+nv}{INOUT}\PYG{p}{)} \PYG{k+kd}{::} \PYG{n+nv}{v12}\PYG{p}{(}\PYG{n+nv}{ndim}\PYG{p}{)} \PYG{c}{! (それ以外)}
\end{Verbatim}
\begin{quote}

入力は残差ベクトル( \code{v2})と行列の積. 出力は,
更新された残差ベクトルの2-ノルムが,
先頭の要素に格納される(これは収束の具合を表示して調べる時などに用いる).
\end{quote}

\begin{Verbatim}[commandchars=\\\{\}]
\PYG{k+kt}{REAL}\PYG{p}{(}\PYG{l+m+mi}{8}\PYG{p}{)}\PYG{p}{,}\PYG{k}{INTENT}\PYG{p}{(}\PYG{n+nv}{INOUT}\PYG{p}{)} \PYG{k+kd}{::} \PYG{n+nv}{v2}\PYG{p}{(}\PYG{n+nv}{ndim}\PYG{p}{)} \PYG{c}{! (\PYGZdq{}CG\PYGZus{}R\PYGZus{}update\PYGZdq{} の場合)}
\PYG{k+kt}{COMPLEX}\PYG{p}{(}\PYG{l+m+mi}{8}\PYG{p}{)}\PYG{p}{,}\PYG{k}{INTENT}\PYG{p}{(}\PYG{n+nv}{INOUT}\PYG{p}{)} \PYG{k+kd}{::} \PYG{n+nv}{v2}\PYG{p}{(}\PYG{n+nv}{ndim}\PYG{p}{)} \PYG{c}{! (それ以外)}
\end{Verbatim}
\begin{quote}

入力は残差ベクトル.
出力は更新された残差ベクトル.
\end{quote}

\begin{Verbatim}[commandchars=\\\{\}]
\PYG{k+kt}{REAL}\PYG{p}{(}\PYG{l+m+mi}{8}\PYG{p}{)}\PYG{p}{,}\PYG{k}{INTENT}\PYG{p}{(}\PYG{n+nv}{IN}\PYG{p}{)} \PYG{k+kd}{::} \PYG{n+nv}{v14}\PYG{p}{(}\PYG{n+nv}{ndim}\PYG{p}{)} \PYG{c}{! (\PYGZdq{}CG\PYGZus{}R\PYGZus{}update\PYGZdq{} の場合)}
\PYG{k+kt}{COMPLEX}\PYG{p}{(}\PYG{l+m+mi}{8}\PYG{p}{)}\PYG{p}{,}\PYG{k}{INTENT}\PYG{p}{(}\PYG{n+nv}{IN}\PYG{p}{)} \PYG{k+kd}{::} \PYG{n+nv}{v14}\PYG{p}{(}\PYG{n+nv}{ndim}\PYG{p}{)} \PYG{c}{! (それ以外)}
\end{Verbatim}
\begin{quote}

影の残差ベクトル( \code{v4})と行列の積.
\end{quote}

\begin{Verbatim}[commandchars=\\\{\}]
\PYG{k+kt}{REAL}\PYG{p}{(}\PYG{l+m+mi}{8}\PYG{p}{)}\PYG{p}{,}\PYG{k}{INTENT}\PYG{p}{(}\PYG{n+nv}{INOUT}\PYG{p}{)} \PYG{k+kd}{::} \PYG{n+nv}{v4}\PYG{p}{(}\PYG{n+nv}{ndim}\PYG{p}{)} \PYG{c}{! (\PYGZdq{}CG\PYGZus{}R\PYGZus{}update\PYGZdq{} の場合)}
\PYG{k+kt}{COMPLEX}\PYG{p}{(}\PYG{l+m+mi}{8}\PYG{p}{)}\PYG{p}{,}\PYG{k}{INTENT}\PYG{p}{(}\PYG{n+nv}{INOUT}\PYG{p}{)} \PYG{k+kd}{::} \PYG{n+nv}{v4}\PYG{p}{(}\PYG{n+nv}{ndim}\PYG{p}{)} \PYG{c}{! (それ以外)}
\end{Verbatim}
\begin{quote}

入力は影の残差ベクトル.
出力は更新された影の残差ベクトル.
\end{quote}

\begin{Verbatim}[commandchars=\\\{\}]
\PYG{k+kt}{INTEGER}\PYG{p}{,}\PYG{k}{INTENT}\PYG{p}{(}\PYG{n+nv}{OUT}\PYG{p}{)} \PYG{k+kd}{::} \PYG{n+nv}{status}\PYG{p}{(}\PYG{l+m+mi}{3}\PYG{p}{)}
\end{Verbatim}
\begin{quote}

エラーコードを返す.

第一成分( \code{status(1)})
\begin{quote}

解が収束した場合,
もしくは計算が破綻した場合には現在の総反復回数に
マイナスが付いた値が返される.
それ以外の場合には現在の総反復回数(マイナスが付かない)が返される.
\code{status(1)} が正の値の時のみ反復を続行できる.
それ以外の場合は反復を進めても有意な結果は得られない.
\end{quote}

第二成分( \code{status(2)})
\begin{quote}

{\hyperref[komega_usage_ja:init]{\emph{*\_init}}} ルーチンで, \code{itermax} を有限にして,
かつ \code{itermax} 回の反復で
収束に達しなかった場合には \code{1} が返される.
\(\alpha\) が発散した場合には \code{2} が返される.
\(\pi_{\rm seed}\) が0にになった場合には \code{3} が返される.
\code{COCG\_update} もしくは \code{BiCG\_update} で,
残差ベクトルと影の残差ベクトルが直交した場合には \code{4} が返される.
それ以外の場合には \code{0} が返される.
\end{quote}

第三成分( \code{status(3)})
\begin{quote}

シード点のindexが返される.
\end{quote}
\end{quote}
\end{quote}


\subsection{*\_getcoef}
\label{komega_usage_ja:getcoef}\label{komega_usage_ja:id6}
後でリスタートをするときに必要な係数を取得する.
このルーチンを呼び出すためには,
{\hyperref[komega_usage_ja:init]{\emph{*\_init}}} ルーチンで \code{itermax} を \code{0} 以外の値にしておく必要がある.

また, このルーチンで使われる総反復回数 (\code{iter\_old}) は {\hyperref[komega_usage_ja:update]{\emph{*\_update}}} の出力 \code{status}
を用いて次のように計算される.

\begin{Verbatim}[commandchars=\\\{\}]
\PYG{n+nv}{iter\PYGZus{}old} \PYG{o}{=} \PYG{n+nb}{ABS}\PYG{p}{(}\PYG{n+nv}{status}\PYG{p}{(}\PYG{l+m+mi}{1}\PYG{p}{)}\PYG{p}{)}
\end{Verbatim}

構文
\begin{quote}

Fortran (シリアル/OpenMP版)

\begin{Verbatim}[commandchars=\\\{\}]
\PYG{k}{CALL }\PYG{n+nv}{komega\PYGZus{}cg\PYGZus{}r\PYGZus{}getcoef}\PYG{p}{(}\PYG{n+nv}{alpha\PYGZus{}save}\PYG{p}{,} \PYG{n+nv}{beta\PYGZus{}save}\PYG{p}{,} \PYG{n+nv}{z\PYGZus{}seed}\PYG{p}{,} \PYG{n+nv}{r\PYGZus{}l\PYGZus{}save}\PYG{p}{)}
\PYG{k}{CALL }\PYG{n+nv}{komega\PYGZus{}cg\PYGZus{}c\PYGZus{}getcoef}\PYG{p}{(}\PYG{n+nv}{alpha\PYGZus{}save}\PYG{p}{,} \PYG{n+nv}{beta\PYGZus{}save}\PYG{p}{,} \PYG{n+nv}{z\PYGZus{}seed}\PYG{p}{,} \PYG{n+nv}{r\PYGZus{}l\PYGZus{}save}\PYG{p}{)}
\PYG{k}{CALL }\PYG{n+nv}{komega\PYGZus{}cocg\PYGZus{}getcoef}\PYG{p}{(}\PYG{n+nv}{alpha\PYGZus{}save}\PYG{p}{,} \PYG{n+nv}{beta\PYGZus{}save}\PYG{p}{,} \PYG{n+nv}{z\PYGZus{}seed}\PYG{p}{,} \PYG{n+nv}{r\PYGZus{}l\PYGZus{}save}\PYG{p}{)}
\PYG{k}{CALL }\PYG{n+nv}{komega\PYGZus{}bicg\PYGZus{}getcoef}\PYG{p}{(}\PYG{n+nv}{alpha\PYGZus{}save}\PYG{p}{,} \PYG{n+nv}{beta\PYGZus{}save}\PYG{p}{,} \PYG{n+nv}{z\PYGZus{}seed}\PYG{p}{,} \PYG{n+nv}{r\PYGZus{}l\PYGZus{}save}\PYG{p}{)}
\end{Verbatim}

Fortran (MPI/ハイブリッド並列版)

\begin{Verbatim}[commandchars=\\\{\}]
\PYG{k}{CALL }\PYG{n+nv}{pkomega\PYGZus{}cg\PYGZus{}r\PYGZus{}getcoef}\PYG{p}{(}\PYG{n+nv}{alpha\PYGZus{}save}\PYG{p}{,} \PYG{n+nv}{beta\PYGZus{}save}\PYG{p}{,} \PYG{n+nv}{z\PYGZus{}seed}\PYG{p}{,} \PYG{n+nv}{r\PYGZus{}l\PYGZus{}save}\PYG{p}{)}
\PYG{k}{CALL }\PYG{n+nv}{pkomega\PYGZus{}cg\PYGZus{}c\PYGZus{}getcoef}\PYG{p}{(}\PYG{n+nv}{alpha\PYGZus{}save}\PYG{p}{,} \PYG{n+nv}{beta\PYGZus{}save}\PYG{p}{,} \PYG{n+nv}{z\PYGZus{}seed}\PYG{p}{,} \PYG{n+nv}{r\PYGZus{}l\PYGZus{}save}\PYG{p}{)}
\PYG{k}{CALL }\PYG{n+nv}{pkomega\PYGZus{}cocg\PYGZus{}getcoef}\PYG{p}{(}\PYG{n+nv}{alpha\PYGZus{}save}\PYG{p}{,} \PYG{n+nv}{beta\PYGZus{}save}\PYG{p}{,} \PYG{n+nv}{z\PYGZus{}seed}\PYG{p}{,} \PYG{n+nv}{r\PYGZus{}l\PYGZus{}save}\PYG{p}{)}
\PYG{k}{CALL }\PYG{n+nv}{pkomega\PYGZus{}bicg\PYGZus{}getcoef}\PYG{p}{(}\PYG{n+nv}{alpha\PYGZus{}save}\PYG{p}{,} \PYG{n+nv}{beta\PYGZus{}save}\PYG{p}{,} \PYG{n+nv}{z\PYGZus{}seed}\PYG{p}{,} \PYG{n+nv}{r\PYGZus{}l\PYGZus{}save}\PYG{p}{)}
\end{Verbatim}

C/C++ (シリアル/OpenMP版)

\begin{Verbatim}[commandchars=\\\{\}]
\PYG{n}{komega\PYGZus{}cg\PYGZus{}r\PYGZus{}getcoef}\PYG{p}{(}\PYG{n}{alpha\PYGZus{}save}\PYG{p}{,} \PYG{n}{beta\PYGZus{}save}\PYG{p}{,} \PYG{o}{\PYGZam{}}\PYG{n}{z\PYGZus{}seed}\PYG{p}{,} \PYG{n}{r\PYGZus{}l\PYGZus{}save}\PYG{p}{)}\PYG{p}{;}
\PYG{n}{komega\PYGZus{}cg\PYGZus{}c\PYGZus{}getcoef}\PYG{p}{(}\PYG{n}{alpha\PYGZus{}save}\PYG{p}{,} \PYG{n}{beta\PYGZus{}save}\PYG{p}{,} \PYG{o}{\PYGZam{}}\PYG{n}{z\PYGZus{}seed}\PYG{p}{,} \PYG{n}{r\PYGZus{}l\PYGZus{}save}\PYG{p}{)}\PYG{p}{;}
\PYG{n}{komega\PYGZus{}cocg\PYGZus{}getcoef}\PYG{p}{(}\PYG{n}{alpha\PYGZus{}save}\PYG{p}{,} \PYG{n}{beta\PYGZus{}save}\PYG{p}{,} \PYG{o}{\PYGZam{}}\PYG{n}{z\PYGZus{}seed}\PYG{p}{,} \PYG{n}{r\PYGZus{}l\PYGZus{}save}\PYG{p}{)}\PYG{p}{;}
\PYG{n}{komega\PYGZus{}bicg\PYGZus{}getcoef}\PYG{p}{(}\PYG{n}{alpha\PYGZus{}save}\PYG{p}{,} \PYG{n}{beta\PYGZus{}save}\PYG{p}{,} \PYG{o}{\PYGZam{}}\PYG{n}{z\PYGZus{}seed}\PYG{p}{,} \PYG{n}{r\PYGZus{}l\PYGZus{}save}\PYG{p}{)}\PYG{p}{;}
\end{Verbatim}

C/C++ (MPI/ハイブリッド並列版)

\begin{Verbatim}[commandchars=\\\{\}]
\PYG{n}{pkomega\PYGZus{}cg\PYGZus{}r\PYGZus{}getcoef}\PYG{p}{(}\PYG{n}{alpha\PYGZus{}save}\PYG{p}{,} \PYG{n}{beta\PYGZus{}save}\PYG{p}{,} \PYG{o}{\PYGZam{}}\PYG{n}{z\PYGZus{}seed}\PYG{p}{,} \PYG{n}{r\PYGZus{}l\PYGZus{}save}\PYG{p}{)}\PYG{p}{;}
\PYG{n}{pkomega\PYGZus{}cg\PYGZus{}c\PYGZus{}getcoef}\PYG{p}{(}\PYG{n}{alpha\PYGZus{}save}\PYG{p}{,} \PYG{n}{beta\PYGZus{}save}\PYG{p}{,} \PYG{o}{\PYGZam{}}\PYG{n}{z\PYGZus{}seed}\PYG{p}{,} \PYG{n}{r\PYGZus{}l\PYGZus{}save}\PYG{p}{)}\PYG{p}{;}
\PYG{n}{pkomega\PYGZus{}cocg\PYGZus{}getcoef}\PYG{p}{(}\PYG{n}{alpha\PYGZus{}save}\PYG{p}{,} \PYG{n}{beta\PYGZus{}save}\PYG{p}{,} \PYG{o}{\PYGZam{}}\PYG{n}{z\PYGZus{}seed}\PYG{p}{,} \PYG{n}{r\PYGZus{}l\PYGZus{}save}\PYG{p}{)}\PYG{p}{;}
\PYG{n}{pkomega\PYGZus{}bicg\PYGZus{}getcoef}\PYG{p}{(}\PYG{n}{alpha\PYGZus{}save}\PYG{p}{,} \PYG{n}{beta\PYGZus{}save}\PYG{p}{,} \PYG{o}{\PYGZam{}}\PYG{n}{z\PYGZus{}seed}\PYG{p}{,} \PYG{n}{r\PYGZus{}l\PYGZus{}save}\PYG{p}{)}\PYG{p}{;}
\end{Verbatim}
\end{quote}

パラメーター
\begin{quote}

\begin{Verbatim}[commandchars=\\\{\}]
\PYG{k+kt}{REAL}\PYG{p}{(}\PYG{l+m+mi}{8}\PYG{p}{)}\PYG{p}{,}\PYG{k}{INTENT}\PYG{p}{(}\PYG{n+nv}{OUT}\PYG{p}{)} \PYG{k+kd}{::} \PYG{n+nv}{alpha\PYGZus{}save}\PYG{p}{(}\PYG{n+nv}{iter\PYGZus{}old}\PYG{p}{)} \PYG{c}{! (\PYGZdq{}CG\PYGZus{}R\PYGZus{}restart\PYGZdq{}, \PYGZdq{}cg\PYGZus{}c\PYGZus{}restart\PYGZdq{}の場合)}
\PYG{k+kt}{COMPLEX}\PYG{p}{(}\PYG{l+m+mi}{8}\PYG{p}{)}\PYG{p}{,}\PYG{k}{INTENT}\PYG{p}{(}\PYG{n+nv}{OUT}\PYG{p}{)} \PYG{k+kd}{::} \PYG{n+nv}{alpha\PYGZus{}save}\PYG{p}{(}\PYG{n+nv}{iter\PYGZus{}old}\PYG{p}{)} \PYG{c}{! (それ以外)}
\end{Verbatim}
\begin{quote}

各反復での(Bi)CG法のパラメーター \(\alpha\).
\end{quote}

\begin{Verbatim}[commandchars=\\\{\}]
\PYG{k+kt}{REAL}\PYG{p}{(}\PYG{l+m+mi}{8}\PYG{p}{)}\PYG{p}{,}\PYG{k}{INTENT}\PYG{p}{(}\PYG{n+nv}{OUT}\PYG{p}{)} \PYG{k+kd}{::} \PYG{n+nv}{beta\PYGZus{}save}\PYG{p}{(}\PYG{n+nv}{iter\PYGZus{}old}\PYG{p}{)} \PYG{c}{! (\PYGZdq{}CG\PYGZus{}R\PYGZus{}restart\PYGZdq{}, \PYGZdq{}cg\PYGZus{}c\PYGZus{}restart\PYGZdq{}の場合)}
\PYG{k+kt}{COMPLEX}\PYG{p}{(}\PYG{l+m+mi}{8}\PYG{p}{)}\PYG{p}{,}\PYG{k}{INTENT}\PYG{p}{(}\PYG{n+nv}{OUT}\PYG{p}{)} \PYG{k+kd}{::} \PYG{n+nv}{beta\PYGZus{}save}\PYG{p}{(}\PYG{n+nv}{iter\PYGZus{}old}\PYG{p}{)} \PYG{c}{! (それ以外)}
\end{Verbatim}
\begin{quote}

各反復での(Bi)CG法のパラメーター \(\beta\).
\end{quote}

\begin{Verbatim}[commandchars=\\\{\}]
\PYG{k+kt}{REAL}\PYG{p}{(}\PYG{l+m+mi}{8}\PYG{p}{)}\PYG{p}{,}\PYG{k}{INTENT}\PYG{p}{(}\PYG{n+nv}{OUT}\PYG{p}{)} \PYG{k+kd}{::} \PYG{n+nv}{z\PYGZus{}seed} \PYG{c}{! (\PYGZdq{}CG\PYGZus{}R\PYGZus{}restart\PYGZdq{}, \PYGZdq{}cg\PYGZus{}c\PYGZus{}restart\PYGZdq{}の場合)}
\PYG{k+kt}{COMPLEX}\PYG{p}{(}\PYG{l+m+mi}{8}\PYG{p}{)}\PYG{p}{,}\PYG{k}{INTENT}\PYG{p}{(}\PYG{n+nv}{OUT}\PYG{p}{)} \PYG{k+kd}{::} \PYG{n+nv}{z\PYGZus{}seed} \PYG{c}{! (それ以外)}
\end{Verbatim}
\begin{quote}

シードシフト.
\end{quote}

\begin{Verbatim}[commandchars=\\\{\}]
\PYG{k+kt}{REAL}\PYG{p}{(}\PYG{l+m+mi}{8}\PYG{p}{)}\PYG{p}{,}\PYG{k}{INTENT}\PYG{p}{(}\PYG{n+nv}{IN}\PYG{p}{)} \PYG{k+kd}{::} \PYG{n+nv}{r\PYGZus{}l\PYGZus{}save}\PYG{p}{(}\PYG{n+nv}{nl}\PYG{p}{,}\PYG{n+nv}{iter\PYGZus{}old}\PYG{p}{)} \PYG{c}{! (\PYGZdq{}CG\PYGZus{}R\PYGZus{}restart\PYGZdq{}の場合)}
\PYG{k+kt}{COMPLEX}\PYG{p}{(}\PYG{l+m+mi}{8}\PYG{p}{)}\PYG{p}{,}\PYG{k}{INTENT}\PYG{p}{(}\PYG{n+nv}{IN}\PYG{p}{)} \PYG{k+kd}{::} \PYG{n+nv}{r\PYGZus{}l\PYGZus{}save}\PYG{p}{(}\PYG{n+nv}{nl}\PYG{p}{,}\PYG{n+nv}{iter\PYGZus{}old}\PYG{p}{)} \PYG{c}{! (それ以外)}
\end{Verbatim}
\begin{quote}

各反復での射影された残差ベクトル.
\end{quote}
\end{quote}


\subsection{*\_getvec}
\label{komega_usage_ja:id7}\label{komega_usage_ja:getvec}
後でリスタートをするときに必要な残差ベクトルを取得する.
このルーチンを呼び出すためには,
{\hyperref[komega_usage_ja:init]{\emph{*\_init}}} ルーチンで \code{itermax} を \code{0} 以外の値にしておく必要がある.

構文
\begin{quote}

Fortran (シリアル/OpenMP版)

\begin{Verbatim}[commandchars=\\\{\}]
\PYG{k}{CALL }\PYG{n+nv}{komega\PYGZus{}cg\PYGZus{}r\PYGZus{}getvec}\PYG{p}{(}\PYG{n+nv}{r\PYGZus{}old}\PYG{p}{)}
\PYG{k}{CALL }\PYG{n+nv}{komega\PYGZus{}cg\PYGZus{}c\PYGZus{}getvec}\PYG{p}{(}\PYG{n+nv}{r\PYGZus{}old}\PYG{p}{)}
\PYG{k}{CALL }\PYG{n+nv}{komega\PYGZus{}cocg\PYGZus{}getvec}\PYG{p}{(}\PYG{n+nv}{r\PYGZus{}old}\PYG{p}{)}
\PYG{k}{CALL }\PYG{n+nv}{komega\PYGZus{}bicg\PYGZus{}getvec}\PYG{p}{(}\PYG{n+nv}{r\PYGZus{}old}\PYG{p}{,} \PYG{n+nv}{r\PYGZus{}tilde\PYGZus{}old}\PYG{p}{)}
\end{Verbatim}

Fortran (MPI/ハイブリッド並列版)

\begin{Verbatim}[commandchars=\\\{\}]
\PYG{k}{CALL }\PYG{n+nv}{pkomega\PYGZus{}cg\PYGZus{}r\PYGZus{}getvec}\PYG{p}{(}\PYG{n+nv}{r\PYGZus{}old}\PYG{p}{)}
\PYG{k}{CALL }\PYG{n+nv}{pkomega\PYGZus{}cg\PYGZus{}c\PYGZus{}getvec}\PYG{p}{(}\PYG{n+nv}{r\PYGZus{}old}\PYG{p}{)}
\PYG{k}{CALL }\PYG{n+nv}{pkomega\PYGZus{}cocg\PYGZus{}getvec}\PYG{p}{(}\PYG{n+nv}{r\PYGZus{}old}\PYG{p}{)}
\PYG{k}{CALL }\PYG{n+nv}{pkomega\PYGZus{}bicg\PYGZus{}getvec}\PYG{p}{(}\PYG{n+nv}{r\PYGZus{}old}\PYG{p}{,} \PYG{n+nv}{r\PYGZus{}tilde\PYGZus{}old}\PYG{p}{)}
\end{Verbatim}

C/C++ (シリアル/OpenMP版)

\begin{Verbatim}[commandchars=\\\{\}]
\PYG{n}{komega\PYGZus{}cg\PYGZus{}r\PYGZus{}getvec}\PYG{p}{(}\PYG{n}{r\PYGZus{}old}\PYG{p}{)}\PYG{p}{;}
\PYG{n}{komega\PYGZus{}cg\PYGZus{}c\PYGZus{}getvec}\PYG{p}{(}\PYG{n}{r\PYGZus{}old}\PYG{p}{)}\PYG{p}{;}
\PYG{n}{komega\PYGZus{}cocg\PYGZus{}getvec}\PYG{p}{(}\PYG{n}{r\PYGZus{}old}\PYG{p}{)}\PYG{p}{;}
\PYG{n}{komega\PYGZus{}bicg\PYGZus{}getvec}\PYG{p}{(}\PYG{n}{r\PYGZus{}old}\PYG{p}{,} \PYG{n}{r\PYGZus{}tilde\PYGZus{}old}\PYG{p}{)}\PYG{p}{;}
\end{Verbatim}

C/C++ (MPI/ハイブリッド並列版)

\begin{Verbatim}[commandchars=\\\{\}]
\PYG{n}{pkomega\PYGZus{}cg\PYGZus{}r\PYGZus{}getvec}\PYG{p}{(}\PYG{n}{r\PYGZus{}old}\PYG{p}{)}\PYG{p}{;}
\PYG{n}{pkomega\PYGZus{}cg\PYGZus{}c\PYGZus{}getvec}\PYG{p}{(}\PYG{n}{r\PYGZus{}old}\PYG{p}{)}\PYG{p}{;}
\PYG{n}{pkomega\PYGZus{}cocg\PYGZus{}getvec}\PYG{p}{(}\PYG{n}{r\PYGZus{}old}\PYG{p}{)}\PYG{p}{;}
\PYG{n}{pkomega\PYGZus{}bicg\PYGZus{}getvec}\PYG{p}{(}\PYG{n}{r\PYGZus{}old}\PYG{p}{,} \PYG{n}{r\PYGZus{}tilde\PYGZus{}old}\PYG{p}{)}\PYG{p}{;}
\end{Verbatim}
\end{quote}

パラメーター
\begin{quote}

\begin{Verbatim}[commandchars=\\\{\}]
\PYG{k+kt}{REAL}\PYG{p}{(}\PYG{l+m+mi}{8}\PYG{p}{)}\PYG{p}{,}\PYG{k}{INTENT}\PYG{p}{(}\PYG{n+nv}{OUT}\PYG{p}{)} \PYG{k+kd}{::} \PYG{n+nv}{r\PYGZus{}old}\PYG{p}{(}\PYG{n+nv}{ndim}\PYG{p}{)} \PYG{c}{! (\PYGZdq{}CG\PYGZus{}R\PYGZus{}getvec\PYGZdq{} の場合)}
\PYG{k+kt}{COMPLEX}\PYG{p}{(}\PYG{l+m+mi}{8}\PYG{p}{)}\PYG{p}{,}\PYG{k}{INTENT}\PYG{p}{(}\PYG{n+nv}{OUT}\PYG{p}{)} \PYG{k+kd}{::} \PYG{n+nv}{r\PYGZus{}old}\PYG{p}{(}\PYG{n+nv}{ndim}\PYG{p}{)} \PYG{c}{! (それ以外)}
\end{Verbatim}
\begin{quote}

先行する計算での最後から2番目の残差ベクトル.
\end{quote}

\begin{Verbatim}[commandchars=\\\{\}]
\PYG{k+kt}{COMPLEX}\PYG{p}{(}\PYG{l+m+mi}{8}\PYG{p}{)}\PYG{p}{,}\PYG{k}{INTENT}\PYG{p}{(}\PYG{n+nv}{OUT}\PYG{p}{)} \PYG{k+kd}{::} \PYG{n+nv}{r\PYGZus{}tilde\PYGZus{}old}\PYG{p}{(}\PYG{n+nv}{ndim}\PYG{p}{)}
\end{Verbatim}
\begin{quote}

\code{BiCG\_getvec} の場合のみ使用.
先行する計算での最後から2番目の影の残差ベクトル.
\end{quote}
\end{quote}


\subsection{*\_getresidual}
\label{komega_usage_ja:getresidual}
各シフト点での残差ベクトルの2-ノルムを取得する.
このルーチンは {\hyperref[komega_usage_ja:init]{\emph{*\_init}}} と {\hyperref[komega_usage_ja:finalize]{\emph{*\_finalize}}} の間の
任意の場所で呼び出すことが出来る. また,
いつ何回呼び出しても最終的な計算結果には影響を与えない.

構文
\begin{quote}

Fortran (シリアル/OpenMP版)

\begin{Verbatim}[commandchars=\\\{\}]
\PYG{k}{CALL }\PYG{n+nv}{komega\PYGZus{}cg\PYGZus{}r\PYGZus{}getresidual}\PYG{p}{(}\PYG{n+nv}{res}\PYG{p}{)}
\PYG{k}{CALL }\PYG{n+nv}{komega\PYGZus{}cg\PYGZus{}c\PYGZus{}getresidual}\PYG{p}{(}\PYG{n+nv}{res}\PYG{p}{)}
\PYG{k}{CALL }\PYG{n+nv}{komega\PYGZus{}cocg\PYGZus{}getresidual}\PYG{p}{(}\PYG{n+nv}{res}\PYG{p}{)}
\PYG{k}{CALL }\PYG{n+nv}{komega\PYGZus{}bicg\PYGZus{}getresidual}\PYG{p}{(}\PYG{n+nv}{res}\PYG{p}{)}
\end{Verbatim}

Fortran (MPI/ハイブリッド並列版)

\begin{Verbatim}[commandchars=\\\{\}]
\PYG{k}{CALL }\PYG{n+nv}{pkomega\PYGZus{}cg\PYGZus{}r\PYGZus{}getresidual}\PYG{p}{(}\PYG{n+nv}{res}\PYG{p}{)}
\PYG{k}{CALL }\PYG{n+nv}{pkomega\PYGZus{}cg\PYGZus{}c\PYGZus{}getresidual}\PYG{p}{(}\PYG{n+nv}{res}\PYG{p}{)}
\PYG{k}{CALL }\PYG{n+nv}{pkomega\PYGZus{}cocg\PYGZus{}getresidual}\PYG{p}{(}\PYG{n+nv}{res}\PYG{p}{)}
\PYG{k}{CALL }\PYG{n+nv}{pkomega\PYGZus{}bicg\PYGZus{}getresidual}\PYG{p}{(}\PYG{n+nv}{res}\PYG{p}{)}
\end{Verbatim}

C/C++ (シリアル/OpenMP版)

\begin{Verbatim}[commandchars=\\\{\}]
\PYG{n}{komega\PYGZus{}cg\PYGZus{}r\PYGZus{}getresidual}\PYG{p}{(}\PYG{n}{res}\PYG{p}{)}\PYG{p}{;}
\PYG{n}{komega\PYGZus{}cg\PYGZus{}c\PYGZus{}getresidual}\PYG{p}{(}\PYG{n}{res}\PYG{p}{)}\PYG{p}{;}
\PYG{n}{komega\PYGZus{}cocg\PYGZus{}getresidual}\PYG{p}{(}\PYG{n}{res}\PYG{p}{)}\PYG{p}{;}
\PYG{n}{komega\PYGZus{}bicg\PYGZus{}getresidual}\PYG{p}{(}\PYG{n}{res}\PYG{p}{)}\PYG{p}{;}
\end{Verbatim}

C/C++ (MPI/ハイブリッド並列版)

\begin{Verbatim}[commandchars=\\\{\}]
\PYG{n}{pkomega\PYGZus{}cg\PYGZus{}r\PYGZus{}getresidual}\PYG{p}{(}\PYG{n}{res}\PYG{p}{)}\PYG{p}{;}
\PYG{n}{pkomega\PYGZus{}cg\PYGZus{}c\PYGZus{}getresidual}\PYG{p}{(}\PYG{n}{res}\PYG{p}{)}\PYG{p}{;}
\PYG{n}{pkomega\PYGZus{}cocg\PYGZus{}getresidual}\PYG{p}{(}\PYG{n}{res}\PYG{p}{)}\PYG{p}{;}
\PYG{n}{pkomega\PYGZus{}bicg\PYGZus{}getresidual}\PYG{p}{(}\PYG{n}{res}\PYG{p}{)}\PYG{p}{;}
\end{Verbatim}
\end{quote}

パラメーター
\begin{quote}

\begin{Verbatim}[commandchars=\\\{\}]
\PYG{k+kt}{COMPLEX}\PYG{p}{(}\PYG{l+m+mi}{8}\PYG{p}{)}\PYG{p}{,}\PYG{k}{INTENT}\PYG{p}{(}\PYG{n+nv}{OUT}\PYG{p}{)} \PYG{k+kd}{::} \PYG{n+nv}{res}\PYG{p}{(}\PYG{n+nv}{nz}\PYG{p}{)}
\end{Verbatim}
\begin{quote}

各シフト点での残差ベクトルの2-ノルム.
\end{quote}
\end{quote}


\subsection{*\_finalize}
\label{komega_usage_ja:id8}\label{komega_usage_ja:finalize}
ライブラリ内部で割りつけた配列のメモリを解放する.

構文
\begin{quote}

Fortran (シリアル/OpenMP版)

\begin{Verbatim}[commandchars=\\\{\}]
\PYG{k}{CALL }\PYG{n+nv}{komega\PYGZus{}cg\PYGZus{}r\PYGZus{}finalize}\PYG{p}{(}\PYG{p}{)}
\PYG{k}{CALL }\PYG{n+nv}{komega\PYGZus{}cg\PYGZus{}c\PYGZus{}finalize}\PYG{p}{(}\PYG{p}{)}
\PYG{k}{CALL }\PYG{n+nv}{komega\PYGZus{}cocg\PYGZus{}finalize}\PYG{p}{(}\PYG{p}{)}
\PYG{k}{CALL }\PYG{n+nv}{komega\PYGZus{}bicg\PYGZus{}finalize}\PYG{p}{(}\PYG{p}{)}
\end{Verbatim}

Fortran (MPI/ハイブリッド並列版)

\begin{Verbatim}[commandchars=\\\{\}]
\PYG{k}{CALL }\PYG{n+nv}{pkomega\PYGZus{}cg\PYGZus{}r\PYGZus{}finalize}\PYG{p}{(}\PYG{p}{)}
\PYG{k}{CALL }\PYG{n+nv}{pkomega\PYGZus{}cg\PYGZus{}c\PYGZus{}finalize}\PYG{p}{(}\PYG{p}{)}
\PYG{k}{CALL }\PYG{n+nv}{pkomega\PYGZus{}cocg\PYGZus{}finalize}\PYG{p}{(}\PYG{p}{)}
\PYG{k}{CALL }\PYG{n+nv}{pkomega\PYGZus{}bicg\PYGZus{}finalize}\PYG{p}{(}\PYG{p}{)}
\end{Verbatim}

C/C++ (シリアル/OpenMP版)

\begin{Verbatim}[commandchars=\\\{\}]
\PYG{n}{komega\PYGZus{}cg\PYGZus{}r\PYGZus{}finalize}\PYG{p}{(}\PYG{p}{)}\PYG{p}{;}
\PYG{n}{komega\PYGZus{}cg\PYGZus{}c\PYGZus{}finalize}\PYG{p}{(}\PYG{p}{)}\PYG{p}{;}
\PYG{n}{komega\PYGZus{}cocg\PYGZus{}finalize}\PYG{p}{(}\PYG{p}{)}\PYG{p}{;}
\PYG{n}{komega\PYGZus{}bicg\PYGZus{}finalize}\PYG{p}{(}\PYG{p}{)}\PYG{p}{;}
\end{Verbatim}

C/C++ (MPI/ハイブリッド並列版)

\begin{Verbatim}[commandchars=\\\{\}]
\PYG{n}{pkomega\PYGZus{}cg\PYGZus{}r\PYGZus{}finalize}\PYG{p}{(}\PYG{p}{)}\PYG{p}{;}
\PYG{n}{pkomega\PYGZus{}cg\PYGZus{}c\PYGZus{}finalize}\PYG{p}{(}\PYG{p}{)}\PYG{p}{;}
\PYG{n}{pkomega\PYGZus{}cocg\PYGZus{}finalize}\PYG{p}{(}\PYG{p}{)}\PYG{p}{;}
\PYG{n}{pkomega\PYGZus{}bicg\PYGZus{}finalize}\PYG{p}{(}\PYG{p}{)}\PYG{p}{;}
\end{Verbatim}
\end{quote}


\section{Shifted BiCGライブラリを使用したソースコードの例}
\label{komega_usage_ja:shifted-bicg}
以下, 代表的な例としてShifted BiCGライブラリの場合の使用方法を記載する.

\begin{Verbatim}[commandchars=\\\{\}]
\PYG{k}{PROGRAM }\PYG{n+nv}{my\PYGZus{}prog}
  \PYG{c}{!}
  \PYG{k}{USE }\PYG{n+nv}{komega\PYGZus{}bicg}\PYG{p}{,} \PYG{n+nv}{ONLY} \PYG{p}{:} \PYG{n+nv}{komega\PYGZus{}bicg\PYGZus{}init}\PYG{p}{,} \PYG{n+nv}{komega\PYGZus{}bicg\PYGZus{}restart}\PYG{p}{,} \PYG{p}{\PYGZam{}}
  \PYG{p}{\PYGZam{}}                       \PYG{n+nv}{komega\PYGZus{}bicg\PYGZus{}update}\PYG{p}{,} \PYG{n+nv}{komega\PYGZus{}bicg\PYGZus{}getcoef}\PYG{p}{,} \PYG{p}{\PYGZam{}}
  \PYG{p}{\PYGZam{}}                       \PYG{n+nv}{komega\PYGZus{}bicg\PYGZus{}getvec}\PYG{p}{,} \PYG{n+nv}{komega\PYGZus{}bicg\PYGZus{}finalize}
  \PYG{k}{USE }\PYG{n+nv}{solve\PYGZus{}cc\PYGZus{}routines}\PYG{p}{,} \PYG{n+nv}{ONLY} \PYG{p}{:} \PYG{n+nv}{input\PYGZus{}size}\PYG{p}{,} \PYG{n+nv}{input\PYGZus{}restart}\PYG{p}{,} \PYG{p}{\PYGZam{}}
  \PYG{p}{\PYGZam{}}                             \PYG{n+nv}{projection}\PYG{p}{,} \PYG{p}{\PYGZam{}}
  \PYG{p}{\PYGZam{}}                             \PYG{n+nv}{hamiltonian\PYGZus{}prod}\PYG{p}{,} \PYG{n+nv}{generate\PYGZus{}system}\PYG{p}{,} \PYG{p}{\PYGZam{}}
  \PYG{p}{\PYGZam{}}                             \PYG{n+nv}{output\PYGZus{}restart}\PYG{p}{,} \PYG{n+nv}{output\PYGZus{}result}
  \PYG{c}{!}
  \PYG{k}{IMPLICIT }\PYG{k}{NONE}
  \PYG{c}{!}
  \PYG{k+kt}{INTEGER}\PYG{p}{,}\PYG{k}{SAVE} \PYG{k+kd}{::} \PYG{p}{\PYGZam{}}
  \PYG{p}{\PYGZam{}} \PYG{n+nv}{ndim}\PYG{p}{,}    \PYG{p}{\PYGZam{}} \PYG{c}{! Size of Hilvert space}
  \PYG{p}{\PYGZam{}} \PYG{n+nv}{nz}\PYG{p}{,}      \PYG{p}{\PYGZam{}} \PYG{c}{! Number of frequencies}
  \PYG{p}{\PYGZam{}} \PYG{n+nv}{nl}\PYG{p}{,}      \PYG{p}{\PYGZam{}} \PYG{c}{! Number of Left vector}
  \PYG{p}{\PYGZam{}} \PYG{n+nv}{itermax}\PYG{p}{,} \PYG{p}{\PYGZam{}} \PYG{c}{! Max. number of iteraction}
  \PYG{p}{\PYGZam{}} \PYG{n+nv}{iter\PYGZus{}old}   \PYG{c}{! Number of iteraction of previous run}
  \PYG{c}{!}
  \PYG{k+kt}{REAL}\PYG{p}{(}\PYG{l+m+mi}{8}\PYG{p}{)}\PYG{p}{,}\PYG{k}{SAVE} \PYG{k+kd}{::} \PYG{p}{\PYGZam{}}
  \PYG{p}{\PYGZam{}} \PYG{n+nv}{threshold} \PYG{c}{! Convergence Threshold}
  \PYG{c}{!}
  \PYG{k+kt}{COMPLEX}\PYG{p}{(}\PYG{l+m+mi}{8}\PYG{p}{)}\PYG{p}{,}\PYG{k}{SAVE} \PYG{k+kd}{::} \PYG{p}{\PYGZam{}}
  \PYG{p}{\PYGZam{}} \PYG{n+nv}{z\PYGZus{}seed} \PYG{c}{! Seed frequency}
  \PYG{c}{!}
  \PYG{k+kt}{COMPLEX}\PYG{p}{(}\PYG{l+m+mi}{8}\PYG{p}{)}\PYG{p}{,}\PYG{k}{ALLOCATABLE}\PYG{p}{,}\PYG{k}{SAVE} \PYG{k+kd}{::} \PYG{p}{\PYGZam{}}
  \PYG{p}{\PYGZam{}} \PYG{n+nv}{z}\PYG{p}{(}\PYG{p}{:}\PYG{p}{)}         \PYG{c}{! (nz): Frequency}
  \PYG{c}{!}
  \PYG{k+kt}{COMPLEX}\PYG{p}{(}\PYG{l+m+mi}{8}\PYG{p}{)}\PYG{p}{,}\PYG{k}{ALLOCATABLE}\PYG{p}{,}\PYG{k}{SAVE} \PYG{k+kd}{::} \PYG{p}{\PYGZam{}}
  \PYG{p}{\PYGZam{}} \PYG{n+nv}{ham}\PYG{p}{(}\PYG{p}{:}\PYG{p}{,}\PYG{p}{:}\PYG{p}{)}\PYG{p}{,} \PYG{p}{\PYGZam{}}
  \PYG{p}{\PYGZam{}} \PYG{n+nv}{rhs}\PYG{p}{(}\PYG{p}{:}\PYG{p}{)}\PYG{p}{,} \PYG{p}{\PYGZam{}}
  \PYG{p}{\PYGZam{}} \PYG{n+nv}{v12}\PYG{p}{(}\PYG{p}{:}\PYG{p}{)}\PYG{p}{,} \PYG{n+nv}{v2}\PYG{p}{(}\PYG{p}{:}\PYG{p}{)}\PYG{p}{,} \PYG{p}{\PYGZam{}} \PYG{c}{! (ndim): Working vector}
  \PYG{p}{\PYGZam{}} \PYG{n+nv}{v14}\PYG{p}{(}\PYG{p}{:}\PYG{p}{)}\PYG{p}{,} \PYG{n+nv}{v4}\PYG{p}{(}\PYG{p}{:}\PYG{p}{)}\PYG{p}{,} \PYG{p}{\PYGZam{}} \PYG{c}{! (ndim): Working vector}
  \PYG{p}{\PYGZam{}} \PYG{n+nv}{r\PYGZus{}l}\PYG{p}{(}\PYG{p}{:}\PYG{p}{)}\PYG{p}{,} \PYG{p}{\PYGZam{}} \PYG{c}{! (nl) : Projeccted residual vector}
  \PYG{p}{\PYGZam{}} \PYG{n+nv}{x}\PYG{p}{(}\PYG{p}{:}\PYG{p}{,}\PYG{p}{:}\PYG{p}{)} \PYG{c}{! (nl,nz) : Projected result}
  \PYG{c}{!}
  \PYG{c}{! Variables for Restart}
  \PYG{c}{!}
  \PYG{k+kt}{COMPLEX}\PYG{p}{(}\PYG{l+m+mi}{8}\PYG{p}{)}\PYG{p}{,}\PYG{k}{ALLOCATABLE}\PYG{p}{,}\PYG{k}{SAVE} \PYG{k+kd}{::} \PYG{p}{\PYGZam{}}
  \PYG{p}{\PYGZam{}} \PYG{n+nv}{alpha}\PYG{p}{(}\PYG{p}{:}\PYG{p}{)}\PYG{p}{,} \PYG{n+nv}{beta}\PYG{p}{(}\PYG{p}{:}\PYG{p}{)} \PYG{c}{! (iter\PYGZus{}old)}
  \PYG{c}{!}
  \PYG{k+kt}{COMPLEX}\PYG{p}{(}\PYG{l+m+mi}{8}\PYG{p}{)}\PYG{p}{,}\PYG{k}{ALLOCATABLE}\PYG{p}{,}\PYG{k}{SAVE} \PYG{k+kd}{::} \PYG{p}{\PYGZam{}}
  \PYG{p}{\PYGZam{}} \PYG{n+nv}{r\PYGZus{}l\PYGZus{}save}\PYG{p}{(}\PYG{p}{:}\PYG{p}{,}\PYG{p}{:}\PYG{p}{)} \PYG{c}{! (nl,iter\PYGZus{}old) Projected residual vectors}
  \PYG{c}{!}
  \PYG{c}{! Variables for Restart}
  \PYG{c}{!}
  \PYG{k+kt}{INTEGER} \PYG{k+kd}{::} \PYG{p}{\PYGZam{}}
  \PYG{p}{\PYGZam{}} \PYG{n+nv}{iter}\PYG{p}{,}    \PYG{p}{\PYGZam{}} \PYG{c}{! Counter for Iteration}
  \PYG{p}{\PYGZam{}} \PYG{n+nv}{status}\PYG{p}{(}\PYG{l+m+mi}{3}\PYG{p}{)}
  \PYG{c}{!}
  \PYG{k+kt}{LOGICAL} \PYG{k+kd}{::} \PYG{p}{\PYGZam{}}
  \PYG{p}{\PYGZam{}} \PYG{n+nv}{restart\PYGZus{}in}\PYG{p}{,} \PYG{p}{\PYGZam{}} \PYG{c}{! If .TRUE., sestart from the previous result}
  \PYG{p}{\PYGZam{}} \PYG{n+nv}{restart\PYGZus{}out}   \PYG{c}{! If .TRUE., save datas for the next run}
  \PYG{c}{!}
  \PYG{c}{! Input Size of vectors, numerical conditions}
  \PYG{c}{!}
  \PYG{k}{CALL }\PYG{n+nv}{input\PYGZus{}size}\PYG{p}{(}\PYG{n+nv}{ndim}\PYG{p}{,}\PYG{n+nv}{nl}\PYG{p}{,}\PYG{n+nv}{nz}\PYG{p}{)}
  \PYG{k}{CALL }\PYG{n+nv}{input\PYGZus{}condition}\PYG{p}{(}\PYG{n+nv}{itermax}\PYG{p}{,}\PYG{n+nv}{threshold}\PYG{p}{,}\PYG{n+nv}{restart\PYGZus{}in}\PYG{p}{,}\PYG{n+nv}{restart\PYGZus{}out}\PYG{p}{)}
  \PYG{c}{!}
  \PYG{k}{ALLOCATE}\PYG{p}{(}\PYG{n+nv}{v12}\PYG{p}{(}\PYG{n+nv}{ndim}\PYG{p}{)}\PYG{p}{,} \PYG{n+nv}{v2}\PYG{p}{(}\PYG{n+nv}{ndim}\PYG{p}{)}\PYG{p}{,} \PYG{n+nv}{v14}\PYG{p}{(}\PYG{n+nv}{ndim}\PYG{p}{)}\PYG{p}{,} \PYG{n+nv}{v4}\PYG{p}{(}\PYG{n+nv}{ndim}\PYG{p}{)}\PYG{p}{,} \PYG{n+nv}{r\PYGZus{}l}\PYG{p}{(}\PYG{n+nv}{nl}\PYG{p}{)}\PYG{p}{,} \PYG{p}{\PYGZam{}}
  \PYG{p}{\PYGZam{}}        \PYG{n+nv}{x}\PYG{p}{(}\PYG{n+nv}{nl}\PYG{p}{,}\PYG{n+nv}{nz}\PYG{p}{)}\PYG{p}{,} \PYG{n+nv}{z}\PYG{p}{(}\PYG{n+nv}{nz}\PYG{p}{)}\PYG{p}{,} \PYG{n+nv}{ham}\PYG{p}{(}\PYG{n+nv}{ndim}\PYG{p}{,}\PYG{n+nv}{ndim}\PYG{p}{)}\PYG{p}{,} \PYG{n+nv}{rhs}\PYG{p}{(}\PYG{n+nv}{ndim}\PYG{p}{)}\PYG{p}{)}
  \PYG{c}{!}
  \PYG{k}{CALL }\PYG{n+nv}{generate\PYGZus{}system}\PYG{p}{(}\PYG{n+nv}{ndim}\PYG{p}{,} \PYG{n+nv}{ham}\PYG{p}{,} \PYG{n+nv}{rhs}\PYG{p}{,} \PYG{n+nv}{z}\PYG{p}{)}
  \PYG{c}{!}
  \PYG{k}{WRITE}\PYG{p}{(}\PYG{o}{*}\PYG{p}{,}\PYG{o}{*}\PYG{p}{)}
  \PYG{k}{WRITE}\PYG{p}{(}\PYG{o}{*}\PYG{p}{,}\PYG{o}{*}\PYG{p}{)} \PYG{l+s+s2}{\PYGZdq{}\PYGZsh{}\PYGZsh{}\PYGZsh{}\PYGZsh{}\PYGZsh{}  CG Initialization  \PYGZsh{}\PYGZsh{}\PYGZsh{}\PYGZsh{}\PYGZsh{}\PYGZdq{}}
  \PYG{k}{WRITE}\PYG{p}{(}\PYG{o}{*}\PYG{p}{,}\PYG{o}{*}\PYG{p}{)}
  \PYG{c}{!}
  \PYG{k}{IF}\PYG{p}{(}\PYG{n+nv}{restart\PYGZus{}in}\PYG{p}{)} \PYG{k}{THEN}
    \PYG{c}{!}
    \PYG{k}{CALL }\PYG{n+nv}{input\PYGZus{}restart}\PYG{p}{(}\PYG{n+nv}{iter\PYGZus{}old}\PYG{p}{,} \PYG{n+nv}{zseed}\PYG{p}{,} \PYG{n+nv}{alpha}\PYG{p}{,} \PYG{n+nv}{beta}\PYG{p}{,} \PYG{n+nv}{r\PYGZus{}l\PYGZus{}save}\PYG{p}{)}
    \PYG{c}{!}
    \PYG{k}{IF}\PYG{p}{(}\PYG{n+nv}{restart\PYGZus{}out}\PYG{p}{)} \PYG{k}{THEN}
\PYG{k}{       }\PYG{k}{CALL }\PYG{n+nv}{komega\PYGZus{}bicg\PYGZus{}restart}\PYG{p}{(} \PYG{p}{\PYGZam{}}
       \PYG{p}{\PYGZam{}}    \PYG{n+nv}{ndim}\PYG{p}{,} \PYG{n+nv}{nl}\PYG{p}{,} \PYG{n+nv}{nz}\PYG{p}{,} \PYG{n+nv}{x}\PYG{p}{,} \PYG{n+nv}{z}\PYG{p}{,} \PYG{n+nv}{itermax}\PYG{p}{,} \PYG{n+nv}{threshold}\PYG{p}{,} \PYG{p}{\PYGZam{}}
       \PYG{p}{\PYGZam{}}    \PYG{n+nv}{status}\PYG{p}{,} \PYG{n+nv}{iter\PYGZus{}old}\PYG{p}{,} \PYG{n+nv}{v2}\PYG{p}{,} \PYG{n+nv}{v12}\PYG{p}{,} \PYG{n+nv}{v4}\PYG{p}{,} \PYG{n+nv}{v14}\PYG{p}{,} \PYG{n+nv}{alpha}\PYG{p}{,} \PYG{p}{\PYGZam{}}
       \PYG{p}{\PYGZam{}}    \PYG{n+nv}{beta}\PYG{p}{,} \PYG{n+nv}{z\PYGZus{}seed}\PYG{p}{,} \PYG{n+nv}{r\PYGZus{}l\PYGZus{}save}\PYG{p}{)}
    \PYG{k}{ELSE}
\PYG{k}{       }\PYG{k}{CALL }\PYG{n+nv}{komega\PYGZus{}bicg\PYGZus{}restart}\PYG{p}{(} \PYG{p}{\PYGZam{}}
       \PYG{p}{\PYGZam{}}    \PYG{n+nv}{ndim}\PYG{p}{,} \PYG{n+nv}{nl}\PYG{p}{,} \PYG{n+nv}{nz}\PYG{p}{,} \PYG{n+nv}{x}\PYG{p}{,} \PYG{n+nv}{z}\PYG{p}{,} \PYG{l+m+mi}{0}\PYG{p}{,} \PYG{n+nv}{threshold}\PYG{p}{,} \PYG{p}{\PYGZam{}}
       \PYG{p}{\PYGZam{}}    \PYG{n+nv}{status}\PYG{p}{,} \PYG{n+nv}{iter\PYGZus{}old}\PYG{p}{,} \PYG{n+nv}{v2}\PYG{p}{,} \PYG{n+nv}{v12}\PYG{p}{,} \PYG{n+nv}{v4}\PYG{p}{,} \PYG{n+nv}{v14}\PYG{p}{,} \PYG{n+nv}{alpha}\PYG{p}{,} \PYG{p}{\PYGZam{}}
       \PYG{p}{\PYGZam{}}    \PYG{n+nv}{beta}\PYG{p}{,} \PYG{n+nv}{z\PYGZus{}seed}\PYG{p}{,} \PYG{n+nv}{r\PYGZus{}l\PYGZus{}save}\PYG{p}{)}
    \PYG{k}{END }\PYG{k}{IF}
    \PYG{c}{!}
    \PYG{c}{! These vectors were saved in BiCG routine}
    \PYG{c}{!}
    \PYG{k}{DEALLOCATE}\PYG{p}{(}\PYG{n+nv}{alpha}\PYG{p}{,} \PYG{n+nv}{beta}\PYG{p}{,} \PYG{n+nv}{r\PYGZus{}l\PYGZus{}save}\PYG{p}{)}
    \PYG{c}{!}
    \PYG{k}{IF}\PYG{p}{(}\PYG{n+nv}{status}\PYG{p}{(}\PYG{l+m+mi}{1}\PYG{p}{)} \PYG{o}{/}\PYG{o}{=} \PYG{l+m+mi}{0}\PYG{p}{)} \PYG{k}{GOTO }\PYG{l+m+mi}{10}
    \PYG{c}{!}
  \PYG{k}{ELSE}
     \PYG{c}{!}
     \PYG{c}{! Generate Right Hand Side Vector}
     \PYG{c}{!}
     \PYG{n+nv}{v2}\PYG{p}{(}\PYG{l+m+mi}{1}\PYG{p}{:}\PYG{n+nv}{ndim}\PYG{p}{)} \PYG{o}{=} \PYG{n+nv}{rhs}\PYG{p}{(}\PYG{l+m+mi}{1}\PYG{p}{:}\PYG{n+nv}{ndim}\PYG{p}{)}
     \PYG{n+nv}{v4}\PYG{p}{(}\PYG{l+m+mi}{1}\PYG{p}{:}\PYG{n+nv}{ndim}\PYG{p}{)} \PYG{o}{=} \PYG{n+nb}{CONJG}\PYG{p}{(}\PYG{n+nv}{v2}\PYG{p}{(}\PYG{l+m+mi}{1}\PYG{p}{:}\PYG{n+nv}{ndim}\PYG{p}{)}\PYG{p}{)}
     \PYG{c}{!v4(1:ndim) = v2(1:ndim)}
     \PYG{c}{!}
     \PYG{k}{IF}\PYG{p}{(}\PYG{n+nv}{restart\PYGZus{}out}\PYG{p}{)} \PYG{k}{THEN}
\PYG{k}{        }\PYG{k}{CALL }\PYG{n+nv}{komega\PYGZus{}bicg\PYGZus{}init}\PYG{p}{(}\PYG{n+nv}{ndim}\PYG{p}{,} \PYG{n+nv}{nl}\PYG{p}{,} \PYG{n+nv}{nz}\PYG{p}{,} \PYG{n+nv}{x}\PYG{p}{,} \PYG{n+nv}{z}\PYG{p}{,} \PYG{n+nv}{termax}\PYG{p}{,} \PYG{n+nv}{threshold}\PYG{p}{)}
     \PYG{k}{ELSE}
\PYG{k}{        }\PYG{k}{CALL }\PYG{n+nv}{komega\PYGZus{}bicg\PYGZus{}init}\PYG{p}{(}\PYG{n+nv}{ndim}\PYG{p}{,} \PYG{n+nv}{nl}\PYG{p}{,} \PYG{n+nv}{nz}\PYG{p}{,} \PYG{n+nv}{x}\PYG{p}{,} \PYG{n+nv}{z}\PYG{p}{,} \PYG{l+m+mi}{0}\PYG{p}{,} \PYG{n+nv}{threshold}\PYG{p}{)}
     \PYG{k}{END }\PYG{k}{IF}
     \PYG{c}{!}
  \PYG{k}{END }\PYG{k}{IF}
  \PYG{c}{!}
  \PYG{c}{! BiCG Loop}
  \PYG{c}{!}
  \PYG{k}{WRITE}\PYG{p}{(}\PYG{o}{*}\PYG{p}{,}\PYG{o}{*}\PYG{p}{)}
  \PYG{k}{WRITE}\PYG{p}{(}\PYG{o}{*}\PYG{p}{,}\PYG{o}{*}\PYG{p}{)} \PYG{l+s+s2}{\PYGZdq{}\PYGZsh{}\PYGZsh{}\PYGZsh{}\PYGZsh{}\PYGZsh{}  CG Iteration  \PYGZsh{}\PYGZsh{}\PYGZsh{}\PYGZsh{}\PYGZsh{}\PYGZdq{}}
  \PYG{k}{WRITE}\PYG{p}{(}\PYG{o}{*}\PYG{p}{,}\PYG{o}{*}\PYG{p}{)}
  \PYG{c}{!}
  \PYG{k}{DO }\PYG{n+nv}{iter} \PYG{o}{=} \PYG{l+m+mi}{1}\PYG{p}{,} \PYG{n+nv}{itermax}
     \PYG{c}{!}
     \PYG{c}{! Projection of Residual vector into the space}
     \PYG{c}{! spaned by left vectors}
     \PYG{c}{!}
     \PYG{n+nv}{r\PYGZus{}l}\PYG{p}{(}\PYG{l+m+mi}{1}\PYG{p}{:}\PYG{n+nv}{nl}\PYG{p}{)} \PYG{o}{=} \PYG{n+nv}{projection}\PYG{p}{(}\PYG{n+nv}{v2}\PYG{p}{(}\PYG{l+m+mi}{1}\PYG{p}{:}\PYG{n+nv}{nl}\PYG{p}{)}\PYG{p}{)}
     \PYG{c}{!}
     \PYG{c}{! Matrix\PYGZhy{}vector product}
     \PYG{c}{!}
     \PYG{k}{CALL }\PYG{n+nv}{hamiltonian\PYGZus{}prod}\PYG{p}{(}\PYG{n+nv}{Ham}\PYG{p}{,} \PYG{n+nv}{v2}\PYG{p}{,} \PYG{n+nv}{v12}\PYG{p}{)}
     \PYG{k}{CALL }\PYG{n+nv}{hamiltonian\PYGZus{}prod}\PYG{p}{(}\PYG{n+nv}{Ham}\PYG{p}{,} \PYG{n+nv}{v4}\PYG{p}{,} \PYG{n+nv}{v14}\PYG{p}{)}
     \PYG{c}{!}
     \PYG{c}{! Update result x with BiCG}
     \PYG{c}{!}
     \PYG{k}{CALL }\PYG{n+nv}{komega\PYGZus{}bicg\PYGZus{}update}\PYG{p}{(}\PYG{n+nv}{v12}\PYG{p}{,} \PYG{n+nv}{v2}\PYG{p}{,} \PYG{n+nv}{v14}\PYG{p}{,} \PYG{n+nv}{v4}\PYG{p}{,} \PYG{n+nv}{x}\PYG{p}{,} \PYG{n+nv}{r\PYGZus{}l}\PYG{p}{,} \PYG{n+nv}{status}\PYG{p}{)}
     \PYG{c}{!}
     \PYG{k}{WRITE}\PYG{p}{(}\PYG{o}{*}\PYG{p}{,}\PYG{l+s+s1}{\PYGZsq{}(a,i,a,3i,a,e15.5)\PYGZsq{}}\PYG{p}{)} \PYG{l+s+s2}{\PYGZdq{}lopp : \PYGZdq{}}\PYG{p}{,} \PYG{n+nv}{iter}\PYG{p}{,} \PYG{p}{\PYGZam{}}
     \PYG{p}{\PYGZam{}}                             \PYG{l+s+s2}{\PYGZdq{}, status : \PYGZdq{}}\PYG{p}{,} \PYG{n+nv}{status}\PYG{p}{(}\PYG{l+m+mi}{1}\PYG{p}{:}\PYG{l+m+mi}{3}\PYG{p}{)}\PYG{p}{,} \PYG{p}{\PYGZam{}}
     \PYG{p}{\PYGZam{}}                             \PYG{l+s+s2}{\PYGZdq{}, Res. : \PYGZdq{}}\PYG{p}{,} \PYG{n+nb}{DBLE}\PYG{p}{(}\PYG{n+nv}{v12}\PYG{p}{(}\PYG{l+m+mi}{1}\PYG{p}{)}\PYG{p}{)}
     \PYG{k}{IF}\PYG{p}{(}\PYG{n+nv}{status}\PYG{p}{(}\PYG{l+m+mi}{1}\PYG{p}{)} \PYG{o}{\PYGZlt{}} \PYG{l+m+mi}{0}\PYG{p}{)} \PYG{k}{EXIT}
     \PYG{c}{!}
  \PYG{k}{END }\PYG{k}{DO}
  \PYG{c}{!}
  \PYG{k}{IF}\PYG{p}{(}\PYG{n+nv}{status}\PYG{p}{(}\PYG{l+m+mi}{2}\PYG{p}{)} \PYG{o}{==} \PYG{l+m+mi}{0}\PYG{p}{)} \PYG{k}{THEN}
\PYG{k}{     }\PYG{k}{WRITE}\PYG{p}{(}\PYG{o}{*}\PYG{p}{,}\PYG{o}{*}\PYG{p}{)} \PYG{l+s+s2}{\PYGZdq{}  Converged in iteration \PYGZdq{}}\PYG{p}{,} \PYG{n+nb}{ABS}\PYG{p}{(}\PYG{n+nv}{status}\PYG{p}{(}\PYG{l+m+mi}{1}\PYG{p}{)}\PYG{p}{)}
  \PYG{k}{ELSE }\PYG{k}{IF}\PYG{p}{(}\PYG{n+nv}{status}\PYG{p}{(}\PYG{l+m+mi}{2}\PYG{p}{)} \PYG{o}{==} \PYG{l+m+mi}{1}\PYG{p}{)} \PYG{k}{THEN}
\PYG{k}{     }\PYG{k}{WRITE}\PYG{p}{(}\PYG{o}{*}\PYG{p}{,}\PYG{o}{*}\PYG{p}{)} \PYG{l+s+s2}{\PYGZdq{}  Not Converged in iteration \PYGZdq{}}\PYG{p}{,} \PYG{n+nb}{ABS}\PYG{p}{(}\PYG{n+nv}{status}\PYG{p}{(}\PYG{l+m+mi}{1}\PYG{p}{)}\PYG{p}{)}
  \PYG{k}{ELSE }\PYG{k}{IF}\PYG{p}{(}\PYG{n+nv}{status}\PYG{p}{(}\PYG{l+m+mi}{2}\PYG{p}{)} \PYG{o}{==} \PYG{l+m+mi}{2}\PYG{p}{)} \PYG{k}{THEN}
\PYG{k}{     }\PYG{k}{WRITE}\PYG{p}{(}\PYG{o}{*}\PYG{p}{,}\PYG{o}{*}\PYG{p}{)} \PYG{l+s+s2}{\PYGZdq{}  Alpha becomes infinity\PYGZdq{}}\PYG{p}{,} \PYG{n+nb}{ABS}\PYG{p}{(}\PYG{n+nv}{status}\PYG{p}{(}\PYG{l+m+mi}{1}\PYG{p}{)}\PYG{p}{)}
  \PYG{k}{ELSE }\PYG{k}{IF}\PYG{p}{(}\PYG{n+nv}{status}\PYG{p}{(}\PYG{l+m+mi}{2}\PYG{p}{)} \PYG{o}{==} \PYG{l+m+mi}{3}\PYG{p}{)} \PYG{k}{THEN}
\PYG{k}{     }\PYG{k}{WRITE}\PYG{p}{(}\PYG{o}{*}\PYG{p}{,}\PYG{o}{*}\PYG{p}{)} \PYG{l+s+s2}{\PYGZdq{}  Pi\PYGZus{}seed becomes zero\PYGZdq{}}\PYG{p}{,} \PYG{n+nb}{ABS}\PYG{p}{(}\PYG{n+nv}{status}\PYG{p}{(}\PYG{l+m+mi}{1}\PYG{p}{)}\PYG{p}{)}
  \PYG{k}{ELSE }\PYG{k}{IF}\PYG{p}{(}\PYG{n+nv}{status}\PYG{p}{(}\PYG{l+m+mi}{2}\PYG{p}{)} \PYG{o}{==} \PYG{l+m+mi}{4}\PYG{p}{)} \PYG{k}{THEN}
\PYG{k}{  }\PYG{k}{WRITE}\PYG{p}{(}\PYG{o}{*}\PYG{p}{,}\PYG{o}{*}\PYG{p}{)} \PYG{l+s+s2}{\PYGZdq{}  Residual \PYGZam{} Shadow residual are orthogonal\PYGZdq{}}\PYG{p}{,} \PYG{p}{\PYGZam{}}
  \PYG{p}{\PYGZam{}}          \PYG{n+nb}{ABS}\PYG{p}{(}\PYG{n+nv}{status}\PYG{p}{(}\PYG{l+m+mi}{1}\PYG{p}{)}\PYG{p}{)}
  \PYG{k}{END }\PYG{k}{IF}
  \PYG{c}{!}
  \PYG{c}{! Total number of iteration}
  \PYG{c}{!}
  \PYG{n+nv}{iter\PYGZus{}old} \PYG{o}{=} \PYG{n+nb}{ABS}\PYG{p}{(}\PYG{n+nv}{status}\PYG{p}{(}\PYG{l+m+mi}{1}\PYG{p}{)}\PYG{p}{)}
  \PYG{c}{!}
  \PYG{c}{! Get these vectors for restart in the Next run}
  \PYG{c}{!}
  \PYG{k}{IF}\PYG{p}{(}\PYG{n+nv}{restart\PYGZus{}out}\PYG{p}{)} \PYG{k}{THEN}
     \PYG{c}{!}
     \PYG{k}{ALLOCATE}\PYG{p}{(}\PYG{n+nv}{alpha}\PYG{p}{(}\PYG{n+nv}{iter\PYGZus{}old}\PYG{p}{)}\PYG{p}{,} \PYG{n+nv}{beta}\PYG{p}{(}\PYG{n+nv}{iter\PYGZus{}old}\PYG{p}{)}\PYG{p}{,} \PYG{n+nv}{r\PYGZus{}l\PYGZus{}save}\PYG{p}{(}\PYG{n+nv}{nl}\PYG{p}{,}\PYG{n+nv}{iter\PYGZus{}old}\PYG{p}{)}\PYG{p}{)}
     \PYG{c}{!}
     \PYG{k}{CALL }\PYG{n+nv}{komega\PYGZus{}bicg\PYGZus{}getcoef}\PYG{p}{(}\PYG{n+nv}{alpha}\PYG{p}{,} \PYG{n+nv}{beta}\PYG{p}{,} \PYG{n+nv}{z\PYGZus{}seed}\PYG{p}{,} \PYG{n+nv}{r\PYGZus{}l\PYGZus{}save}\PYG{p}{)}
     \PYG{k}{CALL }\PYG{n+nv}{komega\PYGZus{}bicg\PYGZus{}getvec}\PYG{p}{(}\PYG{n+nv}{v12}\PYG{p}{,}\PYG{n+nv}{v14}\PYG{p}{)}
     \PYG{c}{!}
     \PYG{k}{CALL }\PYG{n+nv}{output\PYGZus{}restart}\PYG{p}{(}\PYG{n+nv}{iter\PYGZus{}old}\PYG{p}{,} \PYG{n+nv}{z\PYGZus{}seed}\PYG{p}{,} \PYG{n+nv}{alpha}\PYG{p}{,} \PYG{n+nv}{beta}\PYG{p}{,} \PYG{p}{\PYGZam{}}
     \PYG{p}{\PYGZam{}}                   \PYG{n+nv}{r\PYGZus{}l\PYGZus{}save}\PYG{p}{,} \PYG{n+nv}{v12}\PYG{p}{,} \PYG{n+nv}{v14}\PYG{p}{)}
     \PYG{c}{!}
     \PYG{k}{DEALLOCATE}\PYG{p}{(}\PYG{n+nv}{alpha}\PYG{p}{,} \PYG{n+nv}{beta}\PYG{p}{,} \PYG{n+nv}{r\PYGZus{}l\PYGZus{}save}\PYG{p}{)}
     \PYG{c}{!}
  \PYG{k}{END }\PYG{k}{IF}
  \PYG{c}{!}
\PYG{l+m+mi}{10} \PYG{k}{CONTINUE}
  \PYG{c}{!}
  \PYG{c}{! Deallocate all intrinsic vectors}
  \PYG{c}{!}
  \PYG{k}{CALL }\PYG{n+nv}{komega\PYGZus{}bicg\PYGZus{}finalize}\PYG{p}{(}\PYG{p}{)}
  \PYG{c}{!}
  \PYG{c}{! Output to a file}
  \PYG{c}{!}
  \PYG{k}{CALL }\PYG{n+nv}{output\PYGZus{}result}\PYG{p}{(}\PYG{n+nv}{nl}\PYG{p}{,} \PYG{n+nv}{nz}\PYG{p}{,} \PYG{n+nv}{z}\PYG{p}{,} \PYG{n+nv}{x}\PYG{p}{,} \PYG{n+nv}{r\PYGZus{}l}\PYG{p}{)}
  \PYG{c}{!}
  \PYG{k}{DEALLOCATE}\PYG{p}{(}\PYG{n+nv}{v12}\PYG{p}{,} \PYG{n+nv}{v2}\PYG{p}{,} \PYG{n+nv}{v14}\PYG{p}{,} \PYG{n+nv}{v4}\PYG{p}{,} \PYG{n+nv}{r\PYGZus{}l}\PYG{p}{,} \PYG{n+nv}{x}\PYG{p}{,} \PYG{n+nv}{z}\PYG{p}{)}
  \PYG{c}{!}
  \PYG{k}{WRITE}\PYG{p}{(}\PYG{o}{*}\PYG{p}{,}\PYG{o}{*}\PYG{p}{)}
  \PYG{k}{WRITE}\PYG{p}{(}\PYG{o}{*}\PYG{p}{,}\PYG{o}{*}\PYG{p}{)} \PYG{l+s+s2}{\PYGZdq{}\PYGZsh{}\PYGZsh{}\PYGZsh{}\PYGZsh{}\PYGZsh{}  Done  \PYGZsh{}\PYGZsh{}\PYGZsh{}\PYGZsh{}\PYGZsh{}\PYGZdq{}}
  \PYG{k}{WRITE}\PYG{p}{(}\PYG{o}{*}\PYG{p}{,}\PYG{o}{*}\PYG{p}{)}
  \PYG{c}{!}
\PYG{k}{END }\PYG{k}{PROGRAM }\PYG{n+nv}{my\PYGZus{}prog}
\end{Verbatim}


\chapter{Contact}
\label{komega_contact_ja:contact}\label{komega_contact_ja::doc}
このライブラリについてのご意見, ご質問,
バグ報告等ありましたら下記までお問い合わせください。

河村光晶

\begin{Verbatim}[commandchars=\\\{\}]
\PYG{n}{mkawamura\PYGZus{}at\PYGZus{}issp}\PYG{o}{.}\PYG{n}{u}\PYG{o}{\PYGZhy{}}\PYG{n}{tokyo}\PYG{o}{.}\PYG{n}{ac}\PYG{o}{.}\PYG{n}{jp}
\end{Verbatim}

\code{\_at\_}を\code{@}に変えてください.


\chapter{参考文献}
\label{komega_ref_ja:ref}\label{komega_ref_ja::doc}\label{komega_ref_ja:id1}
{[}1{]} A. Frommer, Computing \textbf{70}, 87 (2003).

{[}2{]} S. Yamamoto, T. Sogabe, T. Hoshi, S.-L. Zhang, and T. Fujiwara, J. Phys. Soc. Jpn. \textbf{77}, 114713 (2008).



\renewcommand{\indexname}{索引}
\printindex
\end{document}
